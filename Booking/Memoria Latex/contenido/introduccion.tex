% !TEX encoding = UTF-8 Unicode
% ------------------------------------------------------------------------------
% Este fichero es parte de la plantilla LaTeX para la realización de Proyectos
% Final de Grado, protegido bajo los términos de la licencia GFDL.
% Para más información, la licencia completa viene incluida en el
% fichero fdl-1.3.tex

% Copyright (C) 2012 SPI-FM. Universidad de Cádiz
% ------------------------------------------------------------------------------


\section{Motivación}

\textsl{CoreSport}\footnote{Para más información acerca de CoreSport, visita su página web www.coresport.es} es un centro de mejora del rendimiento y la salud que ofrece, entre otros servicios, clases dirigidas de entrenamiento funcional, TRX, pilates o saco búlgaro, nutrición y dietética, planificación del entrenamiento, fisioterapia, naturopatía, club de atletismo, belleza y bienestar... 
\\
Este proyecto consiste en el desarrollo de una aplicación web que permita la gestión de las actividades del centro, así como de los usuarios del mismo. En concreto, se desarrollará la página web de la empresa, con acceso a área de clientes, donde cada usuario tendrá acceso a la gestión de actividades, así como los administradores a una gestión más amplia sobre actividades y usuarios. 
\\

Actualmente no existe en dicha organización ningún proceso telemático para realizar este tipo de gestión, por lo que todos los datos de actividades y citas quedan registrados en papel. Esto produce mayor esfuerzo para la gestión y el mantenimiento de la información, así como trabajo extra en la comunicación del usuario con el centro para la gestión de sus actividades o citas, ya sea vía telefónica o personalmente en el mismo centro.
\\

Por lo tanto, con el desarrollo del sistema, se ofrecerá una herramienta para ambas partes, administradores y usuarios, que mejorará y facilitará la forma que hasta ahora han tenido para comunicarse y gestionar sus peticiones.


\section{Alcance} 

La aplicación resultante se utilizará vía online por los administradores y usuarios del centro de salud y rendimiento \textsl{CoreSport}, situado en Chiclana de la Frontera. Por lo que será accesible desde cualquier dispositivo con acceso a internet. 
\\

No obstante, aunque el proyecto se centre en los requisitos de esta empresa, se tendrá en cuenta la posibilidad de que otros centros similares hagan uso del software. Por tanto, aspectos claves como las actividades ofrecidas, interfaz de usuario o logotipo de la empresa serán fácilmente adaptables a nuevos posibles centros interesados en el uso de la aplicación. 


\section{Glosario de Términos} 

\begin{itemize} 
\item Entrenamiento funcional: Este tipo de entrenamiento se centra en sesiones cortas, dinámicas, efectivas y entretenidas. Entre otras propiedades, podemos destacar la mejora de movilidad general, tanto articular como muscular, el gran gasto calórico que conlleva o la mejora de habilidades motrices: agilidad, coordinación y equilibrio. 
\item TRX (Entrenamiento en suspensión):  Se considera \textsl{entrenamiento en suspensión} a los ejercicios funcionales que se desarrollan a través de un arnés sujeto por un punto de anclaje, ajustable no elástico fabricado de distintos materiales que permite realizar un entrenamiento completo para todo el cuerpo utilizando el propio peso corporal y la resistencia a la gravedad.
\item Bulgarian Bag (saco búlgaro): equipamiento de ejercicio en forma de luna creciente usado en entrenamiento de fuerza, pliometría, entrenamiento con pesas, ejercicio aeróbico, y fitness en general.
\end {itemize}


\section{Organización del documento}

El presente documento se divide en tres partes bien diferenciadas: 

\begin{itemize} 
\item \textbf{Prolegómeno:} Esta parte contiene una introducción al proyecto, en la cual se explica al lector en qué consistirá este de una forma general junto al contexto donde será usado, además de la planificación del mismo. 
\item La segunda sería la parte de \textbf{desarrollo}, donde se especifican los requisitos, análisis, diseño, construcción y pruebas del sistema. Es decir, se explica en detalle el proceso de desarrollo del proyecto, desde su planteamiento hasta las pruebas realizadas una vez finalizado, incluyendo toda la ingeniería del software. Es la parte más técnica de la documentación.
\item \textbf{Epílogo:} Es la última parte del documento, donde encontraremos principalmente el manual de usuario, bibliografía e información sobre la licencia de la documentación y el software.
 \item \textbf{Software:} El producto final se divide en dos partes: 
 \begin{itemize} 
\item La \textbf{página web} pública de la organización con la que se trabaja, de acceso libre.
\item La \textbf{aplicación web} mediante la cual los administradores y usuarios tendrán la opción de realizar su gestión. Esta se podrá acceder desde la web pública, con la diferencia que se necesitará llevar a cabo un registro para su uso. Sería la parte principal del proyecto.
\end {itemize}
\end {itemize}





