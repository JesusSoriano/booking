% !TEX encoding = UTF-8 Unicode
% ------------------------------------------------------------------------------
% Este fichero es parte de la plantilla LaTeX para la realización de Proyectos
% Final de Grado, protegido bajo los términos de la licencia GFDL.
% Para más información, la licencia completa viene incluida en el
% fichero fdl-1.3.tex

% Copyright (C) 2012 SPI-FM. Universidad de Cádiz
% ------------------------------------------------------------------------------

Las instrucciones de uso del sistema se detallan a continuación.

\section{Introducción}

Este es un sistema de gestión para un centro deportivo, desarrollado concretamente para \textit{CoreSport}, centro para la mejora de la salud y el rendimiento. \\

La aplicación web será accesible desde cualquier dispositivo con conexión a internet y un navegador web, distinguiéndose 3 tipos de usuarios: superadministrador, administrador y usuario. Los socios de la empresa, y trabajadores si se estima oportuno, serán los administradores, mientras que los usuarios serán los clientes del centro. El rol de superadministrador será llevado a cabo por la persona encargada del sistema, en este caso el propio alumno desarrollador del proyecto. 


\section{Características}

Este sistema de gestión proporciona numerosas características que a continuación se detallan: 

\begin{itemize}
\item Los usuarios, administradores y superadministradores podrán realizar las siguientes acciones: 

\begin{itemize}
\item Seleccionar idioma.
\tiem Registrarse en el sistema\footnote{Todo nuevo registro se dará de alta como "usuario", si se tratase de un administrador, será asignado como tal por el superadministrador u otro administrador}.
\tiem Iniciar y cerrar sesión.
\tiem Recuperar la contraseña en caso de olvido.
\tiem Cambiar su contraseña y el resto de sus datos del perfil.
\tiem Mandar y leer correo interno.
\tiem Ver notificaciones del sistema.
\tiem Reservar plaza en una clase y cancelar las reservas.
\tiem Solicitar citas de algún servicio específico, así como cancelar la solicitud o reserva de cita.
\tiem Consultar las reservas realizadas.
\tiem Ver el calendario de actividades con todas las disponibles, las pasadas y las reservadas.
\tiem Ver el histórico de acciones realizadas en el sistema.
\tiem Ver los comunicados.
\tiem Descargarse los archivos a los que tenga acceso. 
\end{itemize}

\item Respecto a administradores y superadministardores, además de estas funcionalidades, podrán: 

\begin{itemize}
\item Responder a solicitudes de cita.
\item Cancelar la cita de un usuario.
\item Activar, suspender y editar usuarios.
\item Activar o suspender a otro administrador.
\item Ver el histórico de acciones de los usuarios del sistema y otros administradores.
\item Dar de alta, editar, suspender o activar servicios.
\item Dar de alta, editar, suspender o activar clase.
\item Dar de alta, editar, suspender o activar cita.
\item Crear nuevo, editar, asignar destinatarios o eliminar archivo.
\item Crear nuevo, editar, suspender o activar comunicado.
\end{itemize}

\item El superadministrador, además, podrá:

\begin{itemize}
\item Activar, suspender o editar administradores.
\end{itemize}

\end{itemize}


\section{Requisitos previos}

Para la utilización del sistema no se requiere ningún elemento hardware o software fuera de lo común. Cualquier dispositivo con conexión a internet y navegador web puede hacer uso de ella. 


\section{Uso del sistema}

El sistema no precisa de conocimiento fuera de lo común en un sistema de gestión. La interfaz es intuitiva y las funcionalidades están estructuradas de manera sencilla a través del menú. Se irá relatando cómo realizar las tareas disponibles, mostrando visualmente algunos ejemplos más representativos. \\

El primer paso para usar la aplicación web sería el registro de usuario. Para ello, 










Describir todos los aspectos necesarios para una utilización efectiva y eficiente del sistema por parte de los usuarios.



