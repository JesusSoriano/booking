% !TEX encoding = UTF-8 Unicode
% ------------------------------------------------------------------------------
% Este fichero es parte de la plantilla LaTeX para la realización de Proyectos
% Final de Grado, protegido bajo los términos de la licencia GFDL.
% Para más información, la licencia completa viene incluida en el
% fichero fdl-1.3.tex

% Copyright (C) 2012 SPI-FM. Universidad de Cádiz
% ------------------------------------------------------------------------------


\documentclass[a4paper,11pt]{book}

% PAQUETES
\usepackage{./estilo/paquetes}
\usepackage{./estilo/colores}
\usepackage{./estilo/comandos}
\usepackage[utf8]{inputenc}

% Ruta al directorio de imágenes
\graphicspath{{./img/}} 

% METADATOS
\title{PLANTILLA PARA TRABAJO FIN DE GRADO}
\author{Profesor Profesor Profesor}
\date{Junio 2012} 
 
\begin{document}

\pagestyle{empty}

% PORTADAS
\input{./portadas/portada-inicial}
\cleardoublepage

% !TEX encoding = UTF-8 Unicode
% ------------------------------------------------------------------------------
% Este fichero es parte de la plantilla LaTeX para la realización de Proyectos
% Final de Grado, protegido bajo los términos de la licencia GFDL.
% Para más información, la licencia completa viene incluida en el
% fichero fdl-1.3.tex

% Copyright (C) 2012 SPI-FM. Universidad de Cádiz
% ------------------------------------------------------------------------------


\begin{center}

  \includegraphics[width=0.3\textwidth]{logo-uca.png} \\

  \vspace{2.5cm}


  \vspace{1.0cm}

  \large{INGENIERÍA INFORMÁTICA} \\

  \vspace{2.0cm}

  \large{GESTIÓN DE CENTRO DE MEJORA DEL RENDIMIENTO Y LA SALUD} \\

  \vspace{2.5cm}

\end{center}

\begin{itemize}
\item \large{Departamento: Ingeniería Informática}
\item \large{Director del proyecto: Lorena Gutiérrez Madroñal}
\item \large{Autor del proyecto: Jesús Soriano Candón}
\end{itemize}

\vspace{0.2cm}

\begin{flushright}
  \large{Cádiz, \today} \\

  \vspace{2.5cm}

  \large{Fdo: Jesús Soriano Candón}
\end{flushright}

\cleardoublepage

% PRELIMINARES
% ------------------------------------------------------------------------------
% Este fichero es parte de la plantilla LaTeX para la realización de Proyectos
% Final de Grado, protegido bajo los términos de la licencia GFDL.
% Para más información, la licencia completa viene incluida en el
% fichero fdl-1.3.tex

% Copyright (C) 2012 SPI-FM. Universidad de Cádiz
% ------------------------------------------------------------------------------

\thispagestyle{empty}

\noindent \textbf{\begin{Large}\textit{Agradecimientos}\end{Large}} 
\newline
\newline
\noindent\textit{Introduzca aqu��, si lo desea, los agradecimientos.}

\newpage

% !TEX encoding = UTF-8 Unicode
% ------------------------------------------------------------------------------
% Este fichero es parte de la plantilla LaTeX para la realización de Proyectos
% Final de Grado, protegido bajo los términos de la licencia GFDL.
% Para más información, la licencia completa viene incluida en el
% fichero fdl-1.3.tex

% Copyright (C) 2012 SPI-FM. Universidad de Cádiz
% ------------------------------------------------------------------------------

\thispagestyle{empty}

\noindent \textbf{\begin{Large}Resumen\end{Large}} 
\newline
\newline
\noindent Este proyecto consiste en el desarrollo de una aplicación web que permita la gestión de las actividades de CoreSport, un centro de mejora del rendimiento y la salud, así como de los usuarios del mismo. En concreto, se desarrollará la página web pública de la empresa, así como el área de clientes, donde cada usuario tendrá acceso a la gestión de actividades, datos personales, mensajería interna, dietas, entrenamientos, etc. A su vez, los administradores del sitio accederán a una gestión más amplia sobre actividades, usuarios y demás características del sistema. Por tanto, este proyecto solventará la necesidad de informatizar la gestión del centro, ofreciendo un mayor control de la misma, a la vez de facilitar el mantenimiento de la información, la comunicación con los usuarios y ofrecerles una herramienta para gestionar cómodamente sus actividades y citas sin necesidad de contactar con el centro para tal fin. \\ 

El sistema se desarrollará bajo Java EE (Enterprise Edition), usando los frameworks JSF (JavaServer Faces), PrimeFaces, EJB (Enterprise JavaBeans) y JPA (Java Persistence API), aparte de los que cada uno de ellos haga uso. \\

La aplicación web se realizará de tal forma que permita una fácil adaptación al uso de la misma por parte de otras instituciones, pudiendo personalizar el estilo y el logo de la interfaz de usuario de acuerdo a las necesidades de cada una de ellas.
\newline

\noindent {\bf Palabras clave:} Gestión de Centro de Mejora del Rendimiento y la Salud, Gestión de Centro Deportivo, Gestión de Gimnasio, CoreSport, Gestión Centro, Gestión Actividades.

\newpage


\frontmatter

% INDICES
\tableofcontents
\listoffigures
\listoftables

\mainmatter

% PROLEGÓMENO
\part{Prolegómeno}
\null\vfill
\noindent La primera parte de la memoria del PFC debe contener una introducción y una planificación del proyecto.\\

La introducción es un capí­tulo que, a modo de resumen, debe contener una breve descripción del contexto de la disciplina en la que el proyecto tiene aplicación y la motivación para su desarrollo, así­ como del alcance previsto.\\

El segundo capí­tulo debe incluir una planificación del proyecto. La planificación deberá ajustarse a las prácticas de ingenierí­a en general, y de la ingenierí­a del software en particular. Deberá tener en cuenta los plazos, los entregables (documentos y software), los recursos (humanos y de equipamiento inventariable) y el mí©todo de ingenierí­a de software a emplear.
\\

\chapter{Introducción}
% !TEX encoding = UTF-8 Unicode
% ------------------------------------------------------------------------------
% Este fichero es parte de la plantilla LaTeX para la realización de Proyectos
% Final de Grado, protegido bajo los términos de la licencia GFDL.
% Para más información, la licencia completa viene incluida en el
% fichero fdl-1.3.tex

% Copyright (C) 2012 SPI-FM. Universidad de Cádiz
% ------------------------------------------------------------------------------


A continuación, se describe la motivación del presente proyecto y su alcance. También se incluye un glosario de términos y la organización del resto de la presente documentación.

\section{Motivación}\label{sec:introduccion}

\textsl{CoreSport}\footnote{Para más información acerca de CoreSport, visita su página web www.coresport.es} es un centro de mejora del rendimiento y la salud que ofrece, entre otros servicios, clases dirigidas de entrenamiento funcional, TRX, nutrición, fisioterapia, saco búlgaro... y otras actividades propias de un centro de estas características.
\\
Este Proyecto Fin de Carrera (PFC) consiste en el desarrollo de una aplicación web que permita la gestión de las actividades del centro, así como de los usuarios del mismo. En concreto, se desarrollará la página web de la empresa, con acceso a área de clientes, donde cada usuario tendrá acceso a la gestión de actividades, así como los administradores a una gestión más amplia sobre actividades y usuarios. 
\\

Actualmente no existe en dicha organización ningún proceso telemático para realizar este tipo de gestión, por lo que todos los datos de actividades y citas quedan registrados en papel. Esto produce mayor esfuerzo para la gestión y el mantenimiento de la información, así como trabajo extra en la comunicación del usuario con el centro para la gestión de sus actividades o citas, ya sea vía telefónica o personalmente en el mismo centro.
\\

Por lo tanto, con el desarrollo del sistema, se ofrecerá una herramienta para ambas partes, administradores y usuarios, que mejorará y facilitará la forma que hasta ahora han tenido para comunicarse y gestionar sus peticiones.


\section{Alcance} 

La aplicación resultante se utilizará vía online por los administradores y usuarios del centro de salud y rendimiento \textsl{CoreSport}, situado en Chiclana de la Frontera. Por lo que será accesible desde cualquier dispositivo con acceso a Internet. 
\\

No obstante, aunque el proyecto se centre en los requisitos de esta empresa, se tendrá en cuenta la posibilidad de que otros centros similares hagan uso del software. Por tanto, aspectos claves como las actividades ofrecidas, interfaz de usuario o logotipo de la empresa serán fácilmente adaptables a nuevos posibles centros interesados en el uso de la aplicación. 


\section{Glosario de Términos} 

\begin{itemize} 
\item Entrenamiento funcional: Este tipo de entrenamiento se centra en sesiones cortas, dinámicas, efectivas y entretenidas. Entre otras propiedades, podemos destacar la mejora de movilidad general, tanto articular como muscular, el gran gasto calórico que conlleva o la mejora de habilidades motrices: agilidad, coordinación y equilibrio. 
\item TRX (Entrenamiento en suspensión):  Se considera \textsl{entrenamiento en suspensión} a los ejercicios funcionales que se desarrollan a través de un arnés sujeto por un punto de anclaje, ajustable no elástico fabricado de distintos materiales que permite realizar un entrenamiento completo para todo el cuerpo utilizando el propio peso corporal y la resistencia a la gravedad.
\item Saco Búlgaro (Bulgarian Bag): Equipamiento de ejercicio en forma de luna creciente usado en entrenamiento de fuerza, pliometría, entrenamiento con pesas, ejercicio aeróbico, y fitness en general.
\end {itemize}


\section{Organización del documento}

El presente documento se divide en tres partes bien diferenciadas: 

\begin{itemize} 
\item \textbf{Prolegómeno:} Esta parte contiene una introducción al proyecto, en la cual se explica al lector en qué consistirá este de una forma general junto al contexto donde será usado, además de la planificación del mismo. 
\item La segunda sería la parte de \textbf{desarrollo}, donde se especifican los requisitos, análisis, diseño, construcción y pruebas del sistema. Es decir, se explica en detalle el proceso de desarrollo del proyecto, desde su planteamiento hasta las pruebas realizadas una vez finalizado, incluyendo toda la ingeniería del software. Es la parte más técnica de la documentación.
\item \textbf{Epílogo:} Es la última parte del documento, donde encontraremos principalmente el manual de usuario, bibliografía e información sobre la licencia de la documentación y el software.
 \item \textbf{Software:} El producto final se divide en dos partes: 
 \begin{itemize} 
\item La \textbf{página web} pública de la organización con la que se trabaja, de acceso libre.
\item La \textbf{aplicación web} mediante la cual los administradores y usuarios tendrán la opción de realizar su gestión. Esta se podrá acceder desde la web pública, con la diferencia que se necesitará llevar a cabo un registro para su uso. Sería la parte principal del proyecto.
\end {itemize}
\end {itemize}







\chapter{Planificación}
% !TEX encoding = UTF-8 Unicode
% ------------------------------------------------------------------------------
% Este fichero es parte de la plantilla LaTeX para la realización de Proyectos
% Final de Grado, protegido bajo los términos de la licencia GFDL.
% Para más información, la licencia completa viene incluida en el
% fichero fdl-1.3.tex

% Copyright (C) 2012 SPI-FM. Universidad de Cádiz
% ------------------------------------------------------------------------------


\section{Metodologí­a de desarrollo}

Previamente al desarrollo de la aplicación web, se ha llevado a cabo el de la web pública de la empresa. Para ello, se ha realizado un diseño inicial de forma orientativa para su posterior desarrollo y pruebas de funcionamiento en diferentes dispositivos. Una vez finalizada, y al poder ser usada independientemente, se ha puesto en producción mientras se implementaba el área de cliente, la parte principal en la que se centrará el proyecto.
\\

Para el desarrollo de la mencionada aplicación web se ha usado un modelo incremental e iterativo. Primeramente se realizó un análisis de la aplicación en general y herramientas a utilizar para su desarrollo. A partir de ahí, se inicia el proceso de implementación del producto divido en varios ciclos, en los cuales se han ido añadiendo distintas funciones a la aplicación, obteniendo así una versión más completa al final de cada una de las fases. En cada una de ellas se analiza, diseña, implementa y prueba estas funcionalidades a añadir.
\\

A continuación, se describirá las tareas realizadas en cada uno de estos ciclos:

\subsection{Primer Ciclo}

Primero de todo, se ha estructurado la implementación del proyecto en distintos módulos. Una vez hecho esto, se ha configurado el servidor de aplicaciones y definido los distintos tipos de usuarios del sistema, implementando seguidamente el registro e identificación de usuarios, con manejo de sesiones y de errores.
\\

A su vez, se ha realizado el diseño general de la interfaz del software, que se ha aplicado a las pantallas de esta fase del desarrollo, como registro de usuario, inicio de sesión o la página de inicio una vez realizado el login, teniendo en cuenta que existen diferentes vistas de la interfaz dependiendo del tipo de usuario. \\
Esta será multilenguaje, dando opción al usuario a elegir su lenguaje por defecto seleccionándolo en el menú desplegable para tal efecto.

\subsection{Segundo Ciclo}

En este segundo ciclo se ha implementado todo lo relacionado con la gestión de usuarios: vista y edición del perfil de los usuarios registrados, gestión de los mismos por parte de administradores, gestión de administradores por el super-administrador, comunicación entre todo tipo de usuario, histórico de acciones en el sistema, etc.

\subsection{Tercer Ciclo}

Este ciclo comprende la principal funcionalidad del sistema, la gestión de actividades y citas. Se ha desarrollado la creación de actividades, grupos, citas, etc., junto a consulta de los mismos, edición, altas, bajas... Siempre desde una interfaz intuitiva que incluye un calendario donde ver toda la actividad del usuario. 
\\

Es la parte más compleja e interesante del producto, ya que cumple el principal requisito funcional por el cual se ha realizado este proyecto.

\subsection{Cuarto Ciclo}

En este último ciclo se han añadido las funcionalidades restantes para la finalización del producto, como notificaciones, contacto o contenido de la página de inicio, por ejemplo. 


\subsection{Quinto Ciclo: Pruebas}

Finalmente, se han realizado las pruebas pertinentes del software y se ha procedido a subsanar los errores encontrados y llevar a cabo pequeñas mejoras. Aclarar que en cada ciclo se han realizado pruebas manuales de la parte correspondiente, por lo que esta fase final de pruebas se ha realizado sin mucha incidencia.


\section{Planificación del proyecto}
Estimación temporal y definición del calendario básico (hitos principales e iteraciones). Desarrollo de la planificación detallada, utilizando un diagrama de Gantt. Los diagramas de Gantt que se vean correctamente (girados y divididos si hace falta).\\

Se debe incluir una comparación cuantitativa del tiempo y el esfuerzo realmente invertido frente al estimado y planificado. Estos datos pueden recogerse del sistema de gestión de tareas empleado para el seguimiento del proyecto.

\section{Organización}

Respecto al desarrollo de la aplicación, he sido el único desarrollador, a la vez de testeador. La tutora del proyecto ha sido Lorena Gutiérrez Madroñal, guiándome en su proceso y asegurándose que cumplía los requisitos suficientes para que sea un proyecto completo. 
\\

En cuanto al cliente, los tres socios de \textsl{CoreSport} con los que he mantenido la comunicación durante el proceso de desarrollo, y posterior a este, han sido Ángel Soriano, Cristina Saucedo y Raúl Coca. Ellos serán los administradores del sistema, con opción de añadir algunos de los trabajadores restantes en el centro, como monitores de las clases, responsables de servicios externos o recepcionista. 
\\

El resto de usuarios del software serán los clientes de la empresa, que serán los usuarios finales del producto mediante previo registro.
\\

El hardware utilizado para el desarrollo ha sido el propio ordenador portátil del alumno, un MacBook Pro, sirviendo así de entorno de programación y pruebas mediante el uso del siguiente software: 

\begin{itemize}
\item \textbf{OS X Yosemite} (versión 10.10.3) como sistema operativo.
\item \textbf{Glassfish} como servidor de aplicaciones local.
\item \textbf{NetBeans} 8.0.2 como entorno de desarrollo.
\item \textbf{pgAdmin3} como gestor y administrador de bases de datos, usando \textbf{PostgreSQL} como sistema de gestión.
\item \textbf{Git} como sistema de control de versiones.
\item \textbf{TeXShop} como editor de textos para la documentación en \textbf{LaTeX}.
\end{itemize}


\section{Costes}

\textbf{TODO: Esta sección se realizará una vez finalizado el proyecto, para comprobar el tiempo empleado y hacer el coste acorde al mismo.}
\\

\textit{Estudio y presupuesto de los costes de los recursos (humanos y materiales) descritos anteriormente, necesarios para el proyecto.}

\textit{Para el cálculo de costes de personal pueden consultarse las tablas salariales de la UCA para el personal técnico de apoyo contratado laboral \cite{paslaboral}, o bien otras más ajustadas a la realidad. El cálculo del coste del personal del proyecto debe hacerse en personas-mes, y luego hacer la correspondencia al coste monetario.}

\section{Riesgos}

Primeramente, tendremos en cuenta el riesgo del hardware. No sabemos cuál va a ser la vida de nuestro equipo en el que estamos desarrollando una aplicación, por lo que siempre debemos tener una copia de nuestro código para evitar posibles pérdidas de datos, ya sea por avería o rotura. Para ello, se ha utilizado \textit{Git}, un sistema de control de versiones que puede ser usado desde nuestro entorno de programación \textit{Netbeans}, brindando una herramienta esencial y sencilla de usar. Tendremos así nuestra implementación en un lugar seguro, además de beneficiarnos de las opciones que un control de versiones ofrece.
\\

\textbf{TODO: Riesgos del software, riesgos durante el desarrollo.}
\\

Una vez el producto esté en producción, su rendimiento dependerá de un servidor contratado, por lo que es un riesgo externo a tener en cuenta. Si el servidor bajo el que la aplicación esté funcionando falla, no se tendrá acceso a la misma. Este es un riesgo que no podemos controlar, y se confía en que la empresa encargada del mismo tome medidas de seguridad suficientes para que, en el caso de fallo, el servicio no sufra caída alguna. 


\section{Aseguramiento de calidad}
En esta sección se incluirán las actividades y tareas relacionadas con el aseguramiento de calidad a realizar durante el desarrollo del software. Se incluirán los estándares, prácticas y normas aplicables durante el desarrollo del software.\\

También, deberán recogerse los diferentes tipos de revisiones, verificaciones y validaciones que se van a llevar a cabo, los criterios para la aceptación o rechazo de cada producto y los procedimientos para implementar acciones correctoras o preventivas.


% DESARROLLO
\part{Desarrollo}
\null\vfill
\noindent En esta parte se debe describir el desarrollo del proyecto siguiendo la metodologí­a empleada. Sus capí­tulos no deben ser una descripción exhaustiva de todos los documentos, diagramas, código fuente y, en general, entregables generados, sino más bien una explicación resumida del desarrollo, estructurada según las etapas principales del proceso de ingenierí­a. Deben seleccionarse aquellos diagramas, fragmentos de código y secciones de los entregables que sean más significativos para dicha explicación. La totalidad de los entregables resultado del proyecto se ubicarán en los anexos y/o en el material en CD/DVD que acompañe al proyecto.

\chapter{Requisitos del Sistema}
% !TEX encoding = UTF-8 Unicode
% ------------------------------------------------------------------------------
% Este fichero es parte de la plantilla LaTeX para la realización de Proyectos
% Final de Grado, protegido bajo los términos de la licencia GFDL.
% Para más información, la licencia completa viene incluida en el
% fichero fdl-1.3.tex

% Copyright (C) 2012 SPI-FM. Universidad de Cádiz
% ------------------------------------------------------------------------------

En esta sección se detalla la situación actual de la organización y las necesidades de la misma, que originan el desarrollo o mejora de un sistema informático. Seguidamente se presentan los objetivos y el catálogo de requisitos del nuevo sistema. Finalmente se describen las tecnologías a usar en la solución al problema.

\section{Situación actual} 

Como se especificó en la sección \ref{sec:introducción}, actualmente \textsl{CoreSport} no consta de proceso telemático alguno para la gestión de usuarios y actividades, por tanto, todos los datos de los servicios que se ofrecen, grupos, citas, etc. quedan registrados en papel con sus respectivas consecuencias, como pueden ser: dificultad para la gestión y el mantenimiento de la información, menor comodidad para el usuario a la hora de obtener información o gestionar sus servicios, mayor tiempo empleado tanto para los administradores del centro como para los usuarios del mismo a la hora de realizar este tipo de gestiones, etc.
\\

La organización dispone, sin embargo, de un software de contabilidad para la gestión de pagos mensuales y puntuales de los usuarios. Este seguirá en uso una vez el producto resultante del PFC entre en producción, por lo que este no incluirá una sección destinada a tal efecto.

\subsection{Procesos de Negocio}\label{subsec:procesosnegocios}

Asimilando que la organización no posee ningún proceso informático para la gestión deseada, analizaremos el proceso seguido para este fin. 
\\

Toda la información de los servicios, usuarios que van a usarlos, horarios, etc., se mantienen en papel. Básicamente, por cada mes se posee un listado de los grupos de actividades colectivas con todos los clientes que pertenecen a ellos, consultando así las plazas libres buscando el grupo correspondiente y apuntando o eliminando usuarios del mismo sobre el papel. \\
Respecto a las citas para el resto de servicios, se gestionan mediante agenda, donde se apunta cada una de ellas, incluyendo servicio, hora y usuario e, igualmente al caso anterior, se consulta si se puede dar cita a una determinada hora para un determinado servicio. 
\\

Como se ha mencionado anteriormente en este documento, la empresa posee un software específico para la gestión de pagos, siendo este el único medio telemático que almacena los datos de usuarios del centro.


\subsection{Entorno Tecnológico}

Al ser un centro pequeño con escasos socios y trabajadores, el entorno tecnológico de la organización es bastante reducido: El ordenador principal está situado en la recepción del centro, el cual posee el software de gestión de pagos mencionado y periféricos básicos junto a un datáfono para cobro de cuotas y servicios. Aparte de esto, varios de los administradores hacen uso de sus propios portátiles como herramienta de trabajo, ya sea para seguimiento de los clientes en algunos servicios, realizar dietas, como herramienta de comunicación, investigación, etc. 
\\

El centro también posee conexión wifi para uso interno.
\\

Asimismo, poseen también una página web realizada a través una plataforma de desarrollo de webs, mediante una de las plantillas que ofrecían.


\subsection{Fortalezas y Debilidades}

Como fortaleza, cabe destacar que se trata de una empresa en crecimiento, con buenos profesionales del sector. Tanto es así, que se han visto obligado a trasladarse a un centro más amplio y con mejores instalaciones que el anterior, debido al incremento de usuarios y de servicios ofertados.
\\

Su principal debilidad podría recaer en la falta de un software para la gestión de usuarios y servicios, situación que genera la creación del producto de este PFC. Decir que la organización del centro y sus trabajadores, usando este tipo de procesos para la información, ha funcionado de manera eficaz hasta el momento.


\section{Necesidades de Negocio}

Los procesos de negocios que la aplicación reflejará son los explicados anteriormente en el punto \ref{subsec:procesosnegocios}, pero en este caso de una forma informatizada, donde los clientes toman también cierto protagonismo a la hora de la gestión tanto de los servicios, como de los propios usuarios.


\section{Objetivos del Sistema}

De acuerdo a lo tratado hasta el momento, los principales objetivos a cumplir serían los siguientes: 

\begin{itemize}
\item Creación de una página web conteniendo información referente a la organización, sus servicios, contacto, etc. 
\item Desarrollo de un sistema online que ofrezca al menos: 
\begin{itemize}
\item Gestión de clientes.
\item Gesión de administradores.
\item Gestión de servicios por parte de clientes y administradores, incluyendo entrenamiento grupal, individual y citas entre otros servicios.
\item Comunicación entre usuarios.
\end {itemize}
\end{itemize}


\section{Catálogo de Requisitos}

\subsection{Requisitos funcionales.}\label{subsec:requisitosfuncionales}

Los requisitos funcionales que debe cumplir el sistema son los siguientes: 

\begin{itemize}
\item Seleccionar idioma.
\item Registro.
\item Login.
\item Restablecer contraseña.
\item Logout.
\item Cambiar contraseña.
\item Modificar datos de usuario.
\item Comunicarse con otros usuarios del sistema.
\item Gestionar clases, citas y otros servicios (alta, baja y modificación).
\item Calendario de actividades.
\item Notificaciones, tales como nuevos emails.
\item Registro de operaciones llevadas a cabo en el sistema.
\item Siguientes opciones para administradores:
\begin{itemize}
\item Gestión de usuarios (suspender, activar y modificar).
\item Gestión de administradores (total para administrador del sistema y limitada para administradores de la organización).
\item Gestión de servicios (alta, baja y modificación).
\end{itemize}
\end{itemize}


\subsection{Requisitos no funcionales}

Los requisitos no funcionales que debe cumplir el producto final del PFC estarán relaciones con la calidad del software, seguridad, etc., especificándose cada uno de ellos a continuación: 

\begin{itemize}
\item Disponibilidad: El sistema deberá estar disponible las 24 horas del día. Esto dependerá del servidor externo donde se aloje el producto, aunque en regla general será un requisito que podrá cumplirse. \\
Será accesible desde cualquier dispositivo con acceso a Internet y opción de navegación.
\item Fiabilidad: Deberá ser una aplicación fiable para todo tipo de usuario, que no presente errores y contenga un mínimo de seguridad, tanto en posibles ataques como en la gestión de la base de datos. Para ello, las contraseñas se almacenarán encriptadas con un algoritmo adecuado y se gestionará el acceso de usuario mediante sesión y funciones habilitadas dependiendo de su rol. 
\item Internacionalización del producto, al menos disponible en español e inglés.
\item Alto grado de usabilidad, ya que el software será utilizado por una gran variedad de perfiles de usuario. Se mantendrá una interfaz intuitiva y de fácil acceso y uso.
\item Mantenibilidad:  Poseerá un fácil mantenimiento del software, ya que prácticamente no requerirá acción alguna. Además, el código tiene en cuenta la escalabilidad del producto y se podrá modificar o añadir funcionalidades de forma cómoda e intuitiva para el desarrollador.  
\end{itemize}


\subsection{Reglas de negocio}

Respecto a las reglas de negocio, la organización especifica simplemente: 

\begin{itemize}
\item El producto final deberá ser de acceso online para estar disponible desde cualquier ubicación en cualquier momento.
\item Hacer visible los términos y condiciones del uso de la aplicación, que pueden ser cambiantes.
\end{itemize}


\subsection{Requisitos de información}

El sistema gestionará datos de usuarios y de los servicios que ofrece la empresa. 
\\
De los usuarios, se recogerán los siguientes campos obligatorios: 

\begin{itemize}
\item Nombre y apellidos.
\item Correo electrónico.
\item Contraseña.
\end{itemize}

Y opcionales:

\begin{itemize}
\item Teléfono.
\item Dirección.
\item Ciudad.
\item País.
\item Código Postal.
\end{itemize}

Respecto a los servicios, se guardarán los siguientes datos: 

\begin{itemize}
\item Nombre del servicio.
\item Descripción.
\item Grupos de usuarios, en caso aplicable / Usuario individual en otros casos.
\item Horarios en los que se ofrece el servicio.
\end{itemize}



\section{Solución Propuesta}

El producto que se desarrollará en este PFC consistirá en una aplicación web donde, tanto administradores como clientes, interactuarán para lograr una gestión óptima de usuarios y servicios. Estará basado en los procesos de negocios explicados en \ref{subsec:procesosnegocios}, con la diferencia que será un proceso informatizado accesible las 24 horas del día desde cualquier localización. 
\\

Cada usuario tendrá acceso a sus datos y a los servicios ofrecidos por la organización para su gestión, así como los administradores podrán realizar acciones similares con la ventaja de poder gestionar no sólo sus datos y los servicios del centro, si no los de cada uno de los clientes, ofreciendo así una aplicación de gestión completa y personalizada, con posibilidad de alta, baja y modificación de usuarios y servicios en tiempo real.
\\

Se hará uso de un servidor de aplicaciones para el alojamiento del producto, así como de una base de datos para el almacenamiento de la información. Ambos serán externos a la organización, contratando uno de tantos servicios ofrecidos en la red que posean las cualidades y seguridad deseada para tal fin. 



\chapter{Análisis del Sistema}
% !TEX encoding = UTF-8 Unicode
% ------------------------------------------------------------------------------
% Este fichero es parte de la plantilla LaTeX para la realización de Proyectos
% Final de Grado, protegido bajo los términos de la licencia GFDL.
% Para más información, la licencia completa viene incluida en el
% fichero fdl-1.3.tex

% Copyright (C) 2012 SPI-FM. Universidad de Cádiz
% ------------------------------------------------------------------------------


Esta sección cubre el análisis del sistema de información a desarrollar, haciendo uso del lenguaje de modelado UML.

\section{Modelo Conceptual}
A partir de los requisitos de información, se desarrollará un diagrama conceptual de clases UML, identificando las clases, atributos, relaciones, restricciones adicionales y reglas de derivación necesarias.



\section{Modelo de Casos de Uso}

A continuación se describirán los casos de uso correspondientes a los requisitos funcionales listados anteriormente en \ref{subsec:requisitosfuncionales}. Estos casos de uso se pueden emplear como mecanismo para representar las interacciones entre los actores y el sistema:
\\

\begin{table}[H]
  \begin{center}
    \begin{tabularx}{16.4cm}{|l|X|}
      \hline
      \textbf{UC-01} & \textbf{Seleccionar idioma}\\
      \hline
      \textbf{Descripción} & Cambiar el idioma en el que se muestra la aplicación.\\
      \hline
      \textbf{Actores} & Usuario.\\
      \hline
      \textbf{Precondiciones} & Ninguna.\\
      \hline
      \textbf{Postcondiciones} & El idioma de la aplicación se establecerá al que el usuario seleccione. Se establecerá este lenguaje por defecto para el usuario.\\
      \hline
      \textbf{Escenario principal} & \smallskip 1. El usuario selecciona uno de los idiomas disponibles en la lista de selección del mismo, en cualquier pantalla de la aplicación.\\
      & 2. La aplicación se mostrará en el idioma seleccionado.\\
      & 3. El idioma se establece por defecto para el resto de veces que el usuario acceda a la aplicación.\\
      & \\
      \hline
      \textbf{Escenarios alternativos} & \\
      & \\
      \hline
    \end{tabularx}
    \caption{UC-01: Seleccionar idioma}
    \label{tab:casousoselidioma}
  \end{center}
\end{table}


\begin{table}[H]
  \begin{center}
    \begin{tabularx}{16.4cm}{|l|X|}
      \hline
      \textbf{UC-02} & \textbf{Registro}\\
      \hline
      \textbf{Descripción} & Registro de usuario en el sistema.\\
      \hline
      \textbf{Actores} & Usuario.\\
      \hline
      \textbf{Precondiciones} & El usuario no puede haberse registrado previamente.\\
      \hline
      \textbf{Postcondiciones} & El usuario quedará registrado en el sistema y se mostrará la página de home del mismo.\\
      \hline
      \textbf{Escenario principal} & \smallskip 1. El sistema muestra la página de registro con los campos correspondientes.\\
      & 2. El usuario rellena al menos los campos obligatorios y hace click en el botón de registro.\\
      & 3. El sistema valida los datos y registra al usuario en la base de datos.\\
      & 4. El sistema muestra la pantalla de aceptación de términos y condiciones.\\
      & 5. El usuario acepta los términos y condiciones.\\
      & 6. El sistema realiza el login del usuario y lo redirige automáticamente a su página de inicio de la aplicación.\\
      & \\
      \hline
      \textbf{Escenarios alternativos} & \smallskip 3.a. Los datos introducidos no son válidos:\\
      & \hspace{0.3cm} 3.a.1. El sistema muestra el mensaje de error correspondiente.\\
      & \hspace{0.3cm} 3.a.2. Vuelve al paso 2.\\
      & 5.a. El usuario no acepta los términos y condiciones.\\
      & \hspace{0.3cm} 5.a.1. El sistema muestra el mensaje de error correspondiente.\\
      & \hspace{0.3cm} 5.a.2. El sistema no permite al usuario acceder al menú principal hasta que los términos y condiciones no se hayan aceptado.\\
      & \hspace{0.3cm} 5.a.3. Vuelve al paso 4.\\
      & \\
      \hline
    \end{tabularx}
    \caption{UC-02: Registro}
  \end{center}
\end{table}


\begin{table}[H]
  \begin{center}
    \begin{tabularx}{16.4cm}{|l|X|}
      \hline
      \textbf{UC-03} & \textbf{Login}\\
      \hline
      \textbf{Descripción} & Inicio de sesión del usuario en el sistema.\\
      \hline
      \textbf{Actores} & Usuario.\\
      \hline
      \textbf{Precondiciones} & Ninguna.\\
      \hline
      \textbf{Postcondiciones} & El usuario se identificará en el sistema y se mostrará la página de home.\\
      \hline
      \textbf{Escenario principal} & \smallskip 1. El sistema muestra la pantalla de inicio de sesión.\\
      & 2. El usuario introduce su correo electrónico y contraseña.\\
      & 3. El sistema valida los datos e inicia la sesión del usuario.\\
      & 4. El sistema muestra la página de home con el menú principal.\\
      & \\
      \hline
      \textbf{Escenarios alternativos} & \smallskip 3.a. Los datos introducidos no son válidos:\\
      & \hspace{0.3cm} 3.a.1. El sistema muestra el mensaje de error correspondiente.\\
      & \hspace{0.3cm} 3.a.2. Vuelve al paso 2.\\
      & \\
      \hline
    \end{tabularx}
    \caption{UC-03: Login}
  \end{center}
\end{table}


\begin{table}[H]
  \begin{center}
    \begin{tabularx}{16.4cm}{|l|X|}
      \hline
      \textbf{UC-04} & \textbf{Restablecer contraseña}\\
      \hline
      \textbf{Descripción} & Restablecer la contraseña del usuario si esta ha sido olvidada.\\
      \hline
      \textbf{Actores} & Usuario.\\
      \hline
      \textbf{Precondiciones} & El usuario debe estar registrado en el sistema y su correo electrónico estar operativo.\\
      \hline
      \textbf{Postcondiciones} & El usuario establecerá una nueva contraseña para su login.\\
      \hline
      \textbf{Escenario principal} & \smallskip 1. El sistema muestra la pantalla de restablecer contraseña. \\
      & 2. El usuario introduce la dirección de correo electrónico que usó para su registro.\\
      & 3. El sistema valida la dirección y envía un correo con el enlace para restablecer la contraseña.\\
      & 4. El usuario accede al enlace recibido en el email.\\
      & 5. El usuario introduce su nueva contraseña y la repite por seguridad.\\
      & 6. El sistema valida los datos y la nueva contraseña para el usuario queda registrada en la base de datos.\\
      & \\
      \hline
      \textbf{Escenarios alternativos} & \smallskip  3.a. La dirección de correo introducida no es válida:\\
      & \hspace{0.3cm} 3.a.1. El sistema muestra el mensaje de error correspondiente.\\
      & \hspace{0.3cm} 3.a.2. Vuelve al paso 2.\\
      & 4.a. El correo electrónico no ha sido recibido:\\
      & \hspace{0.3cm} 4.a.1. Vuelve al paso 1.\\
      & 4.b. El enlace no es válido o ha expirado:\\
      & \hspace{0.3cm} 4.b.1. Vuelve al paso 1.\\
      & 6.a. Las contraseñas introducidas no son válidas:\\
      & \hspace{0.3cm} 6.a.1. El sistema muestra el mensaje de error correspondiente.\\
      & \hspace{0.3cm} 6.a.2. Vuelve al paso 5.\\
      & \\
      \hline
    \end{tabularx}
    \caption{UC-04: Recuperar contraseña}
  \end{center}
\end{table}


\begin{table}[H]
  \begin{center}
    \begin{tabularx}{16.4cm}{|l|X|}
      \hline
      \textbf{UC-05} & \textbf{Logout}\\
      \hline
      \textbf{Descripción} & Cerrar la sesión del usuario en el sistema.\\
      \hline
      \textbf{Actores} & Usuario.\\
      \hline
      \textbf{Precondiciones} & Debe ser un usuario previamente identificado.\\
      \hline
      \textbf{Postcondiciones} & Se terminará la sesión del usuario.\\
      \hline
      \textbf{Escenario principal} & \smallskip 1. El usuario hará click en la opción para salir del sistema desde cualquier pantalla de la aplicación.\\
      & 2. El sistema cerrará la sesión, quedando esta inhabilitada.\\
      & \\
      \hline
      \textbf{Escenarios alternativos} & \smallskip 1.a. El usuario hace click en salir del sistema a través del menú de opciones.\\
      & \\
      \hline
    \end{tabularx}
    \caption{UC-05: Logout}
  \end{center}
\end{table}


\begin{table}[H]
  \begin{center}
    \begin{tabularx}{16.4cm}{|l|X|}
      \hline
      \textbf{UC-06} & \textbf{Cambiar contraseña}\\
      \hline
      \textbf{Descripción} & Establecer una nueva contraseña para el usuario.\\
      \hline
      \textbf{Actores} & Usuario.\\
      \hline
      \textbf{Precondiciones} & El usuario debe haberse identificado previamente.\\
      \hline
      \textbf{Postcondiciones} & El usuario establecerá una nueva contraseña para su login.\\
      \hline
      \textbf{Escenario principal} & \smallskip 1. El sistema muestra la pantalla donde ingresar los datos para la nueva contraseña.\\
      & 2. El usuario introduce los datos pedidos.\\
      & 3. El sistema valida los datos y la nueva contraseña para el usuario queda registrada en la base de datos.\\
      & \\
      \hline
      \textbf{Escenarios alternativos} & \smallskip 3.a. Los datos introducidos no son válidos:\\
      & \hspace{0.3cm} 3.a.1. El sistema muestra el mensaje de error correspondiente.\\
      & \hspace{0.3cm} 3.a.2. Vuelve al paso 2.\\
      & \\
      \hline
    \end{tabularx}
    \caption{UC-06: Cambiar contraseña}
  \end{center}
\end{table}


\begin{table}[H]
  \begin{center}
    \begin{tabularx}{16.4cm}{|l|X|}
      \hline
      \textbf{UC-07} & \textbf{Leer el correo interno.}\\
      \hline
      \textbf{Descripción} & Leer alguno de los correos recibidos o enviados.\\
      \hline
      \textbf{Actores} & Usuario.\\
      \hline
      \textbf{Precondiciones} & El usuario debe haberse identificado previamente.\\
      \hline
      \textbf{Postcondiciones} & El sistema mostrará el mensaje seleccionado.\\
      \hline
      \textbf{Escenario principal} & \smallskip 1. El usuario navega hasta la página deseada donde se encuentra el correo a leer, ya sea recibido o enviado.\\
      & 2. El usuario hace click en el asunto del mensaje en cuestión.\\
      & 3. El sistema muestra el mensaje y sus datos correspondientes. Si el mensaje no había sido previamente abierto, se marcará como mensaje leído.\\
      & \\
      \hline
      \textbf{Escenarios alternativos} & \\
      & \\
      \hline
    \end{tabularx}
    \caption{UC-07: Leer correo interno}
  \end{center}
\end{table}


\begin{table}[H]
  \begin{center}
    \begin{tabularx}{16.4cm}{|l|X|}
      \hline
      \textbf{UC-08} & \textbf{Mandar un correo}\\
      \hline
      \textbf{Descripción} & Mandar un correo interno a otro usuario de la aplicación.\\
      \hline
      \textbf{Actores} & Usuario.\\
      \hline
      \textbf{Precondiciones} & El usuario debe haberse identificado previamente.\\
      \hline
      \textbf{Postcondiciones} & Se mandará el mensaje redactado al usuario seleccionado.\\
      \hline
      \textbf{Escenario principal} & \smallskip 1. El sistema muestra la página para redactar un nuevo email.\\
      & 2. El usuario selecciona destinatario e introduce asunto y el mensaje a mandar.\\
      & 3. El sistema valida los datos y manda el correo a la persona seleccionada.\\
      & \\
      \hline
      \textbf{Escenarios alternativos} & \smallskip 3.a. Los datos introducidos no son válidos:\\
      & \hspace{0.3cm} 3.a.1. El sistema muestra el mensaje de error correspondiente.\\
      & \hspace{0.3cm} 3.a.2. Vuelve al paso 2.\\
      & \\
      \hline
    \end{tabularx}
    \caption{UC-08: Mandar un correo}
  \end{center}
\end{table}


\begin{table}[H]
  \begin{center}
    \begin{tabularx}{16.4cm}{|l|X|}
      \hline
      \textbf{UC-09} & \textbf{Alta en un grupo}\\
      \hline
      \textbf{Descripción} & El usuario se dará de alta en un grupo de un determinado servicio ofrecido.\\
      \hline
      \textbf{Actores} & Usuario y administrador o superadministrador.\\
      \hline
      \textbf{Precondiciones} & El usuario y administrador deben haberse identificado previamente.\\
      \hline
      \textbf{Postcondiciones} & El usuario será dado de alta en el grupo deseado.\\
      \hline
      \textbf{Escenario principal} & \smallskip 1. El sistema muestra los grupos disponibles.\\
      & 2. El usuario selecciona el grupo a ingresar haciendo click en la opción disponible para unirse al mismo. \\
      & 3. El sistema manda una petición de unión al grupo a los administradores.\\
      & 4. El administrador recibe la notificación de petición y se dirige a la página de grupos.\\
      & 5. El administrador acepta la petición del usuario.\\
      & 6. El sistema registra al usuario en el grupo, quedando una plaza menos libre, y manda una notificación de aceptación al usuario.\\
      & \\
      \hline
      \textbf{Escenarios alternativos} & \smallskip 5.a. El administrador declina la petición del usuario:\\
      & \hspace{0.3cm} 5.a.1. El sistema envía una notificación al usuario.\\
      & \hspace{0.3cm} 5.a.2. Vuelve al punto 1.\\
      & \\
      \hline
    \end{tabularx}
    \caption{UC-09: Gestión de servicios: Alta en un grupo}
  \end{center}
\end{table}


\begin{table}[H]
  \begin{center}
    \begin{tabularx}{16.4cm}{|l|X|}
      \hline
      \textbf{UC-10} & \textbf{Baja de un grupo}\\
      \hline
      \textbf{Descripción} & El usuario se dará de baja en un grupo de un determinado servicio ofrecido.\\
      \hline
      \textbf{Actores} & Usuario.\\
      \hline
      \textbf{Precondiciones} & El usuario debe haberse identificado previamente.\\
      \hline
      \textbf{Postcondiciones} & El usuario será dado de baja en el grupo deseado.\\
      \hline
      \textbf{Escenario principal} & \smallskip 1. El sistema mostrará los grupos a los que el usuario pertenece.\\
      & 2. El usuario hará click en la opción correspondiente a darse de baja del grupo deseado.\\
      & 3. El sistema dará de baja al usuario, quedando así su plaza libre en el grupo. \\
      & 4. Los administradores recibirán una notificación de baja del grupo por parte del usuario.\\
      & \\
      \hline
      \textbf{Escenarios alternativos} & \\
      & \\
      \hline
    \end{tabularx}
    \caption{UC-10: Gestión de servicios: Baja de un grupo}
  \end{center}
\end{table}


\begin{table}[H]
  \begin{center}
    \begin{tabularx}{16.4cm}{|l|X|}
      \hline
      \textbf{UC-11} & \textbf{Alta en una clase}\\
      \hline
      \textbf{Descripción} & El usuario podrá darse de alta en una clase puntual de un determinado grupo sin la obligación de pertenecer al mismo.\\
      \hline
      \textbf{Actores} & Usuario.\\
      \hline
      \textbf{Precondiciones} & El usuario debe haberse identificado previamente.\\
      \hline
      \textbf{Postcondiciones} & El usuario se dará de alta en una clase puntual de un determinado grupo.\\
      \hline
      \textbf{Escenario principal} & \smallskip 1. El sistema muestra los servicios disponibles.\\
      & 2. El usuario seleccionará el servicio para el cual desea darse de alta en la clase.\\
      & 3. El sistema muestra los grupos disponibles.\\
      & 4. El usuario selecciona el grupo al que pertenece la clase. \\
      & 5. El sistema muestra las próximas clases del grupo.\\
      & 6. El usuario selecciona la clase a ingresar.\\
      & 7. El sistema muestra la información de la clase y las plazas disponibles.\\
      & 8. El usuario elige la opción de registrarse en la clase.\\
      & 9. El sistema añade al usuario a la clase, quedando una plaza libre menos.\\
      & \\
      \hline
      \textbf{Escenarios alternativos} & \smallskip 8.a. La clase a ingresar no dispone de plazas libres.\\
      & \hspace{0.3cm} 8.a.1: Vuelve al punto 6. \\
      & \\
      \hline
    \end{tabularx}
    \caption{UC-11: Gestión de servicios: Alta en una clase}
  \end{center}
\end{table}


\begin{table}[H]
  \begin{center}
    \begin{tabularx}{16.4cm}{|l|X|}
      \hline
      \textbf{UC-12} & \textbf{Baja de una clase}\\
      \hline
      \textbf{Descripción} & El usuario se dará de baja en una clase de su grupo asignado, quedando así su plaza libre ese día puntual.\\
      \hline
      \textbf{Actores} & Usuario.\\
      \hline
      \textbf{Precondiciones} & El usuario debe haberse identificado previamente.\\
      \hline
      \textbf{Postcondiciones} & El usuario se dará de baja en una clase de su grupo asignado.\\
      \hline
      \textbf{Escenario principal} & \smallskip 1. El sistema mostrará los grupos a los que el usuario pertenece.\\
      & 2. El usuario selecciona el grupo al que pertenece la clase. \\
      & 3. El sistema muestra las próximas clases del grupo.\\
      & 4. El usuario selecciona la clase a darse de baja.\\
      & 5. El sistema muestra la información de la clase y las plazas disponibles.\\
      & 6. El usuario hará click en la opción correspondiente para darse de baja de esa clase puntual.\\
      & 7. El sistema dará de baja al usuario, quedando así su plaza libre en la clase. \\
      & \\
      \hline
      \textbf{Escenarios alternativos} & \\
      & \\
      \hline
    \end{tabularx}
    \caption{UC-12: Gestión de servicios: Baja de una clase}
  \end{center}
\end{table}


\begin{table}[H]
  \begin{center}
    \begin{tabularx}{16.4cm}{|l|X|}
      \hline
      \textbf{UC-13} & \textbf{Alta en una clase para un invitado}\\
      \hline
      \textbf{Descripción} & El administrador podrá dar de alta en una clase puntual de un determinado grupo a un usuario no registrado.\\
      \hline
      \textbf{Actores} & Administrador o superadministrador.\\
      \hline
      \textbf{Precondiciones} & El usuario debe haberse identificado previamente como administrador o superadministrador.\\
      \hline
      \textbf{Postcondiciones} & El administrador dará de alta a un invitado en una clase puntual de un determinado grupo.\\
      \hline
      \textbf{Escenario principal} & \smallskip 1. El sistema muestra los servicios disponibles.\\
      & 2. El administrador seleccionará el servicio para el cual desea dar de alta al invitado.\\
      & 3. El sistema muestra los grupos disponibles.\\
      & 4. El administrador selecciona el grupo al que pertenece la clase. \\
      & 5. El sistema muestra las próximas clases del grupo.\\
      & 6. El administrador selecciona la clase a ingresar al usuario no registrado.\\
      & 7. El sistema muestra la información de la clase y las plazas disponibles.\\
      & 8. El administrador elige la opción de registrar un invitado en la clase.\\
      & 9. El sistema añade al usuario no registrado a la clase, quedando una plaza libre menos.\\
      & \\
      \hline
      \textbf{Escenarios alternativos} & \smallskip 8.a. La clase a ingresar no dispone de plazas libres.\\
      & \hspace{0.3cm} 8.a.1: Vuelve al punto 6. \\
      & \\
      \hline
    \end{tabularx}
    \caption{UC-13: Gestión de servicios: Alta en una clase para un invitado}
  \end{center}
\end{table}


\begin{table}[H]
  \begin{center}
    \begin{tabularx}{16.4cm}{|l|X|}
      \hline
      \textbf{UC-14} & \textbf{Baja de una clase para un invitado}\\
      \hline
      \textbf{Descripción} & El administrador podrá dar de baja a un usuario no registrado en una clase puntual a la que se haya añadido previamente. \\
      \hline
      \textbf{Actores} & Administrador o superadministrador..\\
      \hline
      \textbf{Precondiciones} & El usuario debe haberse identificado previamente como administrador o superadministrador. La clase debe tener al menos un invitado registrado.\\
      \hline
      \textbf{Postcondiciones} & El usuario se dará de baja en una clase de su grupo asignado.\\
      \hline
      \textbf{Escenario principal} & \smallskip 1. El sistema muestra los servicios disponibles.\\
      & 2. El administrador seleccionará el servicio para el cual desea dar de baja al invitado.\\
      & 3. El sistema muestra los grupos disponibles.\\
      & 4. El administrador selecciona el grupo al que pertenece la clase. \\
      & 5. El sistema muestra las próximas clases del grupo.\\
      & 6. El administrador selecciona la clase a dar de baja al invitado.\\
      & 7. El sistema muestra la información de la clase y las plazas disponibles.\\
      & 8. El administrador hará click en la opción correspondiente para dar de baja al usuario no registrado de esa clase puntual.\\
      & 9. El sistema dará de baja a dicho usuario, quedando así su plaza libre en la clase. \\
      & \\
      \hline
      \textbf{Escenarios alternativos} & \\
      & \\
      \hline
    \end{tabularx}
    \caption{UC-14: Gestión de servicios: Baja de una clase para un invitado}
  \end{center}
\end{table}


\begin{table}[H]
  \begin{center}
    \begin{tabularx}{16.4cm}{|l|X|}
      \hline
      \textbf{UC-15} & \textbf{Pedir cita}\\
      \hline
      \textbf{Descripción} & El usuario podrá solicitar cita de un determinado servicio a la hora seleccionada.\\
      \hline
      \textbf{Actores} & Usuario y administrador o superadministrador.\\
      \hline
      \textbf{Precondiciones} & El usuario y administrador deben haberse identificado previamente.\\
      \hline
      \textbf{Postcondiciones} & Se le asignará la cita seleccionada al usuario.\\
      \hline
      \textbf{Escenario principal} & \smallskip 1. El sistema muestra los servicios disponibles.\\
      & 2. El usuario seleccionará el servicio para el cual desea solicitar la cita.\\
      & 3. El usuario seleccionará el día en el cual pedir la cita.\\
      & 4. El sistema muestra los rangos de horarios disponibles.\\
      & 5. El usuario selecciona el horario y envía la solicitud.\\
      & 6. El sistema hace llegar la solicitud a los administradores.\\
      & 7. El administrador recibe la notificación de cita.\\
      & 8. El administrador acepta la solicitud del usuario.\\
      & 9. El sistema registra la cita, dejando de estar disponible el horario especificado para el resto de usuarios, y manda una notificación de aceptación al usuario.\\
      & \\
      \hline
      \textbf{Escenarios alternativos} & \smallskip 8.a. El administrador declina la solicitud de cita del usuario:\\
      & \hspace{0.3cm} 8.a.1. El sistema envía una notificación al usuario.\\
      & \hspace{0.3cm} 8.a.2. Vuelve al punto 1.\\
      & \\
      \hline
    \end{tabularx}
    \caption{UC-15: Gestión de servicios: Pedir cita}
  \end{center}
\end{table}


\begin{table}[H]
  \begin{center}
    \begin{tabularx}{16.4cm}{|l|X|}
      \hline
      \textbf{UC-16} & \textbf{Cancelar cita}\\
      \hline
      \textbf{Descripción} & Cancelación de una cita por parte del administrador. El usuario debe contactar con algún administrador para pedir la cancelación de la misma.\\
      \hline
      \textbf{Actores} & Administrador.\\
      \hline
      \textbf{Precondiciones} & El usuario debe haberse identificado previamente como administrador.\\
      \hline
      \textbf{Postcondiciones} & La cita del usuario quedará cancelada por parte del administrador.\\
      \hline
      \textbf{Escenario principal} & \smallskip 1. El sistema muestra el listado de citas.\\
      & 2. El administrador selecciona la cita que desea cancelar.\\
      & 3. El sistema muestra la información de la cita y opciones disponibles.\\
      & 4. El administrador selecciona cancelar la cita.\\
      & 5. El sistema cancela la cita dejando el horario libre nuevamente para las citas de ese determinado servicio, y notifica al usuario.\\
      & \\
      \hline
      \textbf{Escenarios alternativos} & \\
      & \\
      \hline
    \end{tabularx}
    \caption{UC-16: Gestión de servicios: Cancelar cita}
  \end{center}
\end{table}


\begin{table}[H]
  \begin{center}
    \begin{tabularx}{16.4cm}{|l|X|}
      \hline
      \textbf{UC-17} & \textbf{Pedir cita para usuario no registrado}\\
      \hline
      \textbf{Descripción} & El administrador podrá solicitar cita para un usuario no registrado de un determinado servicio a la hora seleccionada.\\
      \hline
      \textbf{Actores} & Administrador o superadministrador.\\
      \hline
      \textbf{Precondiciones} & El usuario deben haberse identificado previamente como administrador o superadministrador..\\
      \hline
      \textbf{Postcondiciones} & Se asignará la cita seleccionada a un usuario no registrado en el sistema.\\
      \hline
      \textbf{Escenario principal} & \smallskip 1. El sistema muestra los servicios disponibles.\\
      & 2. El administrador seleccionará el servicio para el cual desea registrar la cita.\\
      & 3. El administrador seleccionará el día en el cual ocurrirá la cita.\\
      & 4. El sistema muestra los rangos de horarios disponibles.\\
      & 5. El administrador selecciona el horario y la opción de asignación a un invitado.\\
      & 9. El sistema registra la cita, dejando de estar disponible el horario especificado para el resto de usuarios.\\
      & \\
      \hline
      \textbf{Escenarios alternativos} & \\
      & \\
      \hline
    \end{tabularx}
    \caption{UC-17: Gestión de servicios: Pedir cita para usuario no registrado}
  \end{center}
\end{table}


\begin{table}[H]
  \begin{center}
    \begin{tabularx}{16.4cm}{|l|X|}
      \hline
      \textbf{UC-18} & \textbf{Cancelar cita de usuario no registrado}\\
      \hline
      \textbf{Descripción} & Cancelación de una cita por parte del administrador. El usuario debe contactar con algún administrador para pedir la cancelación de la misma.\\
      \hline
      \textbf{Actores} & Administrador o superadministrador..\\
      \hline
      \textbf{Precondiciones} & El usuario debe haberse identificado previamente como administrador o superadministrador..\\
      \hline
      \textbf{Postcondiciones} & La cita del usuario quedará cancelada por parte del administrador.\\
      \hline
      \textbf{Escenario principal} & \smallskip 1. El sistema muestra el listado de citas.\\
      & 2. El administrador selecciona la cita que desea cancelar.\\
      & 3. El sistema muestra la información de la cita y opciones disponibles.\\
      & 4. El administrador selecciona cancelar la cita.\\
      & 5. El sistema cancela la cita dejando el horario libre nuevamente para las citas de ese determinado servicio.\\
      & \\
      \hline
      \textbf{Escenarios alternativos} & \\
      & \\
      \hline
    \end{tabularx}
    \caption{UC-18: Gestión de servicios: Cancelar cita de usuario no registrado}
  \end{center}
\end{table}


\begin{table}[H]
  \begin{center}
    \begin{tabularx}{16.4cm}{|l|X|}
      \hline
      \textbf{UC-19} & \textbf{Calendario de actividades}\\
      \hline
      \textbf{Descripción} & El usuario dispondrá de un calendario donde ver todas las clases y citas. \\
      \hline
      \textbf{Actores} & Usuario.\\
      \hline
      \textbf{Precondiciones} & El usuario debe haberse identificado previamente.\\
      \hline
      \textbf{Postcondiciones} & El sistema mostrará el calendario con las clases y citas del usuario.\\
      \hline
      \textbf{Escenario principal} & \smallskip 1. El usuario se navegará hasta la página correspondiente del calendario.\\
      & 2. El sistema mostrará el calendario con los datos de actividades y citas de todos los servicios que el usuario haya reservado previamente.\\
      & \\
      \hline
      \textbf{Escenarios alternativos} & \\
      & \\
      \hline
    \end{tabularx}
    \caption{UC-19: Calendario de actividades}
  \end{center}
\end{table}


\begin{table}[H]
  \begin{center}
    \begin{tabularx}{16.4cm}{|l|X|}
      \hline
      \textbf{UC-20} & \textbf{Notificaciones}\\
      \hline
      \textbf{Descripción} & El sistema notificará acciones como nuevo correo recibido. \\
      \hline
      \textbf{Actores} & Usuario.\\
      \hline
      \textbf{Precondiciones} & El usuario debe haberse identificado previamente.\\
      \hline
      \textbf{Postcondiciones} & El usuario dispondrá de notificaciones de determinadas acciones del sistema. \\
      \hline
      \textbf{Escenario principal} & \smallskip 1. El usuario se dirigirá a la página correspondiente del calendario.\\
      & 2. El sistema mostrará las últimas notificaciones por orden cronológico, marcando como vistas las nuevas si existiesen.\\
      & \\
      \hline
      \textbf{Escenarios alternativos} & \\
      & \\
      \hline
    \end{tabularx}
    \caption{UC-20: Notificaciones}
  \end{center}
\end{table}


\begin{table}[H]
  \begin{center}
    \begin{tabularx}{16.4cm}{|l|X|}
      \hline
      \textbf{UC-21} & \textbf{Auditoría}\\
      \hline
      \textbf{Descripción} & Registro de operaciones llevadas a cabo en el sistema por el usuario.\\
      \hline
      \textbf{Actores} & Usuario.\\
      \hline
      \textbf{Precondiciones} & Debe ser un usuario previamente identificado.\\
      \hline
      \textbf{Postcondiciones} & El sistema mostrará las acciones llevadas a cabo por el usuario en las fechas indicadas.\\
      \hline
      \textbf{Escenario principal} & \smallskip 1. El sistema mostrará la página correspondiente al histórico de acciones.\\
      & 2. El usuario seleccionará el periodo del que desea ver las acciones llevadas a cabo y el tipo de acción si lo desea.\\
      & 3. El sistema muestra las acciones correspondientes.\\
      & \\
      \hline
      \textbf{Escenarios alternativos} & \\
      & \\
      \hline
    \end{tabularx}
    \caption{UC-21: Auditoría}
  \end{center}
\end{table}


\begin{table}[H]
  \begin{center}
    \begin{tabularx}{16.4cm}{|l|X|}
      \hline
      \textbf{UC-22} & \textbf{Auditoría para administradores}\\
      \hline
      \textbf{Descripción} & Registro de operaciones llevadas a cabo en el sistema por los usuario.\\
      \hline
      \textbf{Actores} & Administrador o superadministrador.\\
      \hline
      \textbf{Precondiciones} & Debe ser un usuario previamente identificado como administrador o superadministrador.\\
      \hline
      \textbf{Postcondiciones} & El sistema mostrará las acciones llevadas a cabo por el usuario seleccionado en las fechas indicadas.\\
      \hline
      \textbf{Escenario principal} & \smallskip 1. El sistema mostrará la página correspondiente al histórico de acciones.\\
      & 2. El administrador seleccionará el periodo del que desea ver las acciones llevadas a cabo, así como el usuario y el tipo de acción si lo desea.\\
      & 3. El sistema muestra las acciones correspondientes.\\
      & \\
      \hline
      \textbf{Escenarios alternativos} & \\
      & \\
      \hline
    \end{tabularx}
    \caption{UC-22: Auditoría para administradores}
  \end{center}
\end{table}


\begin{table}[H]
  \begin{center}
    \begin{tabularx}{16.4cm}{|l|X|}
      \hline
      \textbf{UC-23} & \textbf{Gestión de usuarios: ver, suspender, activar o mandar email}\\
      \hline
      \textbf{Descripción} & Gestión de usuarios, con posibilidad de ver, suspender, activar o mandar email a usuarios.\\
      \hline
      \textbf{Actores} & Administrador o superadministrador.\\
      \hline
      \textbf{Precondiciones} & Debe ser un usuario previamente identificado como administrador o superadministrador.\\
      \hline
      \textbf{Postcondiciones} & El administrador podrá ver, suspender, activar o mandar un email al usuario seleccionado.\\
      \hline
      \textbf{Escenario principal} & \smallskip 1. El sistema muestra el listado de usuarios del sistema.\\
      & 2. El administrador navega por la lista y selecciona la acción correspondiente en la casilla del usuario en cuestión.\\
      & 3. El sistema ejecuta la acción (activar/suspender) o redirecciona al usuario a la página correspondiente para realizar la acción deseada (perfil del usuario o redactar email).\\
      & \\
      \hline
      \textbf{Escenarios alternativos} & \\
      & \\
      \hline
    \end{tabularx}
    \caption{UC-23: Gestión de usuarios: ver, suspender, activar o mandar email}
  \end{center}
\end{table}


\begin{table}[H]
  \begin{center}
    \begin{tabularx}{16.4cm}{|l|X|}
      \hline
      \textbf{UC-24} & \textbf{Gestión de usuarios: editar usuario}\\
      \hline
      \textbf{Descripción} & Modificación de los datos de usuarios.\\
      \hline
      \textbf{Actores} & Administrador o superadministrador.\\
      \hline
      \textbf{Precondiciones} & Debe ser un usuario previamente identificado como administrador o superadministrador.\\
      \hline
      \textbf{Postcondiciones} & El administrador podrá modificar los datos del usuario seleccionado.\\
      \hline
      \textbf{Escenario principal} & \smallskip 1. El administrador se dirige al perfil del usuario deseado siguiendo los pasos de gestión de usuario detallados en el caso de uso anterior.\\
      & 2. El administrador hace click en la opción de modificar usuario. \\
      & 3. El sistema muestra los datos del usuario en campos editables.\\
      & 4. El administrador modifica o rellena los datos correspondientes.\\
      & 5. El sistema modifica los datos y notifica al usuario de los cambios realizados.\\ 
      & \\
      \hline
      \textbf{Escenarios alternativos} & \smallskip 5.a. Los datos introducidos no son válidos:\\
      & \hspace{0.3cm} 5.a.1. El sistema muestra el mensaje de error correspondiente.\\
      & \hspace{0.3cm} 5.a.2. Vuelve al paso 4.\\
      & \\
      \hline
    \end{tabularx}
    \caption{UC-24: Gestión de usuarios: editar usuario}
  \end{center}
\end{table}


\begin{table}[H]
  \begin{center}
    \begin{tabularx}{16.4cm}{|l|X|}
      \hline
      \textbf{UC-25} & \textbf{Ver administrador}\\
      \hline
      \textbf{Descripción} & Gestión de administradores: ver datos de los administradores. \\
      \hline
      \textbf{Actores} & Administrador o superadministrador.\\
      \hline
      \textbf{Precondiciones} & Debe ser un usuario previamente identificado como administrador o superadministrador.\\
      \hline
      \textbf{Postcondiciones} & El administrador podrá ver los datos de otro administrador.\\
      \hline
      \textbf{Escenario principal} & \smallskip 1. El sistema muestra el listado de administradores del sistema.\\
      & 2. El administrador navega por la lista y selecciona la acción correspondiente en la casilla del administrador en cuestión.\\
      & 3. El sistema redirecciona al administrador a la página correspondiente al perfil del usuario.\\
      & \\
      \hline
      \textbf{Escenarios alternativos} & \\
      & \\
      \hline
    \end{tabularx}
    \caption{UC-25: Gestión de administradores: ver datos de los administradores}
  \end{center}
\end{table}


\begin{table}[H]
  \begin{center}
    \begin{tabularx}{16.4cm}{|l|X|}
      \hline
      \textbf{UC-26} & \textbf{Mandar email a administradores}\\
      \hline
      \textbf{Descripción} & Gestión de administradores: mandar correo. \\
      \hline
      \textbf{Actores} & Administrador o superadministrador.\\
      \hline
      \textbf{Precondiciones} & Debe ser un usuario previamente identificado como administrador o superadministrador.\\
      \hline
      \textbf{Postcondiciones} & El administrador podrá ver, activar, suspender o mandar un email al usuario seleccionado.\\
      \hline
      \textbf{Escenario principal} & \smallskip 1. El sistema muestra el listado de administradores del sistema.\\
      & 2. El administrador navega por la lista y selecciona la acción correspondiente en la casilla del administrador en cuestión.\\
      & 3. El sistema redirecciona al administrador a la página correspondiente para redactar un email al administrador seleccionado.\\
      & \\
      \hline
      \textbf{Escenarios alternativos} & \\
      & \\
      \hline
    \end{tabularx}
    \caption{UC-26: Gestión de administradores: mandar correo}
  \end{center}
\end{table}


\begin{table}[H]
  \begin{center}
    \begin{tabularx}{16.4cm}{|l|X|}
      \hline
      \textbf{UC-27} & \textbf{Suspender/activar administrador}\\
      \hline
      \textbf{Descripción} & Gestión de administradores: suspender/activar administrador. \\
      \hline
      \textbf{Actores} & Superadministrador.\\
      \hline
      \textbf{Precondiciones} & Debe ser un usuario previamente identificado como superadministrador.\\
      \hline
      \textbf{Postcondiciones} & El superadministrador podrá ver, activar, suspender o mandar un email al usuario seleccionado.\\
      \hline
      \textbf{Escenario principal} & \smallskip 1. El sistema muestra el listado de administradores del sistema.\\
      & 2. El superadministrador navega por la lista y selecciona la acción correspondiente en la casilla del administrador en cuestión.\\
      & 3. El sistema ejecuta la acción (activar/suspender) y refleja el nuevo estado del administrador en la lista. \\
      & \\
      \hline
      \textbf{Escenarios alternativos} & \\
      & \\
      \hline
    \end{tabularx}
    \caption{UC-27: Gestión de administradores: suspender/activar administrador}
  \end{center}
\end{table}


\begin{table}[H]
  \begin{center}
    \begin{tabularx}{16.4cm}{|l|X|}
      \hline
      \textbf{UC-28} & \textbf{Editar administrador}\\
      \hline
      \textbf{Descripción} & Gestión de administradores: Modificación de los datos de administradores.\\
      \hline
      \textbf{Actores} & Superadministrador.\\
      \hline
      \textbf{Precondiciones} & Debe ser un usuario previamente identificado como superadministrador.\\
      \hline
      \textbf{Postcondiciones} & El superadministrador podrá modificar los datos del administrador seleccionado.\\
      \hline
      \textbf{Escenario principal} & \smallskip 1. El superadministrador se dirige al perfil del usuario deseado siguiendo los pasos de gestión de administrador detallados en el caso de uso correspondiente.\\
      & 2. El superadministrador hace click en la opción de editar. \\
      & 3. El sistema muestra los datos del administrador en campos editables.\\
      & 4. El superadministrador modifica o rellena los datos correspondientes.\\
      & 5. El sistema modifica los datos y notifica al administrador de los cambios realizados.\\ 
      & \\
      \hline
      \textbf{Escenarios alternativos} & \smallskip 5.a. Los datos introducidos no son válidos:\\
      & \hspace{0.3cm} 5.a.1. El sistema muestra el mensaje de error correspondiente.\\
      & \hspace{0.3cm} 5.a.2. Vuelve al paso 4.\\
      & \\
      \hline
    \end{tabularx}
    \caption{UC-28: Gestión de administradores: editar administrador}
  \end{center}
\end{table}


\begin{table}[H]
  \begin{center}
    \begin{tabularx}{16.4cm}{|l|X|}
      \hline
      \textbf{UC-29} & \textbf{Alta de servicio}\\
      \hline
      \textbf{Descripción} & Añadir un servicio nuevo al sistema.\\
      \hline
      \textbf{Actores} & Administrador o superadministrador.\\
      \hline
      \textbf{Precondiciones} & Debe ser un usuario previamente identificado como administrador o superadministrador.\\
      \hline
      \textbf{Postcondiciones} & Se añadirá un nuevo servicio al sistema.\\
      \hline
      \textbf{Escenario principal} & \smallskip 1. El sistema muestra la lista de servicios actuales.\\
      & 2. El administrador hace click en la opción de añadir un nuevo servicio.\\
      & 3. El sistema muestra una ventana con los datos necesarios para la creación del servicio.\\
      & 4. El administrador introduce los datos necesarios.\\
      & 5. El sistema valida los datos y registra el nuevo servicio, quedando reflejado en la lista.\\
      & \\
      \hline
      \textbf{Escenarios alternativos} & \smallskip 5.a. Los datos introducidos no son válidos:\\
      & \hspace{0.3cm} 5.a.1. El sistema muestra el mensaje de error correspondiente.\\
      & \hspace{0.3cm} 5.a.2. Vuelve al paso 4.\\
      & \\
      \hline
    \end{tabularx}
    \caption{UC-29: Gestión de servicios: Alta de servicio}
  \end{center}
\end{table}


\begin{table}[H]
  \begin{center}
    \begin{tabularx}{16.4cm}{|l|X|}
      \hline
      \textbf{UC-30} & \textbf{Suspender/activar servicio}\\
      \hline
      \textbf{Descripción} & Suspender o activar uno de los servicios del sistema.\\
      \hline
      \textbf{Actores} & Administrador o superadministrador.\\
      \hline
      \textbf{Precondiciones} & Debe ser un usuario previamente identificado como administrador o superadministrador.\\
      \hline
      \textbf{Postcondiciones} & El administrador activará o suspenderá unos de los servicios existentes en el sistema.\\
      \hline
      \textbf{Escenario principal} & \smallskip 1. El sistema muestra la lista de servicios actuales.\\
      & 2. El administrador hace click en la opción deseada (suspender/activar) de la casilla del servicio deseado.\\
      & 3. El sistema suspende/activa el servicio y notifica a los usuarios que estén haciendo uso del mismo sobre la acción.\\
      & \\
      \hline
      \textbf{Escenarios alternativos} & \smallskip \\
      & \\
      \hline
    \end{tabularx}
    \caption{UC-30: Gestión de servicios: Suspender/activar servicio}
  \end{center}
\end{table}


\begin{table}[H]
  \begin{center}
    \begin{tabularx}{16.4cm}{|l|X|}
      \hline
      \textbf{UC-31} & \textbf{Editar servicio}\\
      \hline
      \textbf{Descripción} & Editar uno de los servicios del sistema.\\
      \hline
      \textbf{Actores} & Administrador o superadministrador.\\
      \hline
      \textbf{Precondiciones} & Debe ser un usuario previamente identificado como administrador o superadministrador.\\
      \hline
      \textbf{Postcondiciones} & El administrador editará los datos de unos de los servicios existentes en el sistema.\\
      \hline
      \textbf{Escenario principal} & \smallskip 1. El sistema muestra la lista de servicios actuales.\\
      & 2. El administrador hace click en la opción de editar de la casilla del servicio deseado.\\
      & 3. El sistema muestra una ventana editable con los datos actuales del servicio.\\
      & 4. El administrador edita los datos deseados.\\
      & 5. El sistema valida los datos y registra los cambios, quedando reflejados en la lista de servicios.\\
      & \\
      \hline
      \textbf{Escenarios alternativos} & \smallskip 5.a. Los datos introducidos no son válidos:\\
      & \hspace{0.3cm} 5.a.1. El sistema muestra el mensaje de error correspondiente.\\
      & \hspace{0.3cm} 5.a.2. Vuelve al paso 4.\\
      & \\
      \hline
    \end{tabularx}
    \caption{UC-31: Gestión de servicios: Editar servicio}
  \end{center}
\end{table}


\begin{table}[H]
  \begin{center}
    \begin{tabularx}{16.4cm}{|l|X|}
      \hline
      \textbf{UC-32} & \textbf{Alta de grupo}\\
      \hline
      \textbf{Descripción} & Añadir un grupo nuevo de un determinado servicio al sistema. A partir de este momento, los usuarios podrán darse de alta en la actividad.\\
      \hline
      \textbf{Actores} & Administrador o superadministrador.\\
      \hline
      \textbf{Precondiciones} & Debe ser un usuario previamente identificado como administrador o superadministrador.\\
      \hline
      \textbf{Postcondiciones} & Se añadirá un nuevo grupo del servicio seleccionado al sistema.\\
      \hline
      \textbf{Escenario principal} & \smallskip 1. El sistema muestra la lista de grupos actuales.\\
      & 2. El administrador hace click en la opción de añadir un nuevo grupo.\\
      & 3. El sistema muestra una ventana con los datos necesarios para la creación del grupo.\\
      & 4. El administrador introduce los datos necesarios.\\
      & 5. El sistema valida los datos y registra el nuevo grupo, quedando reflejado en la lista.\\
      & \\
      \hline
      \textbf{Escenarios alternativos} & \smallskip 5.a. Los datos introducidos no son válidos:\\
      & \hspace{0.3cm} 5.a.1. El sistema muestra el mensaje de error correspondiente.\\
      & \hspace{0.3cm} 5.a.2. Vuelve al paso 4.\\
      & \\
      \hline
    \end{tabularx}
    \caption{UC-32: Gestión de servicios: Alta de grupo}
  \end{center}
\end{table}


\begin{table}[H]
  \begin{center}
    \begin{tabularx}{16.4cm}{|l|X|}
      \hline
      \textbf{UC-33} & \textbf{Suspender/activar grupo}\\
      \hline
      \textbf{Descripción} & Suspender o activar uno de los grupos de un determinado servicio del sistema.\\
      \hline
      \textbf{Actores} & Administrador o superadministrador.\\
      \hline
      \textbf{Precondiciones} & Debe ser un usuario previamente identificado como administrador o superadministrador.\\
      \hline
      \textbf{Postcondiciones} & El administrador activará o suspenderá unos de los grupos existentes en el sistema.\\
      \hline
      \textbf{Escenario principal} & \smallskip 1. El sistema muestra la lista de grupos actuales.\\
      & 2. El administrador hace click en la opción deseada (suspender/activar) de la casilla del grupo deseado.\\
      & 3. El sistema suspende/activa el grupo y notifica a los usuarios que estén haciendo uso del mismo sobre la acción.\\
      & \\
      \hline
      \textbf{Escenarios alternativos} & \smallskip \\
      & \\
      \hline
    \end{tabularx}
    \caption{UC-33: Gestión de servicios: Suspender/activar grupo}
  \end{center}
\end{table}


\begin{table}[H]
  \begin{center}
    \begin{tabularx}{16.4cm}{|l|X|}
      \hline
      \textbf{UC-34} & \textbf{Editar grupo}\\
      \hline
      \textbf{Descripción} & Editar uno de los grupos del sistema.\\
      \hline
      \textbf{Actores} & Administrador o superadministrador.\\
      \hline
      \textbf{Precondiciones} & Debe ser un usuario previamente identificado como administrador o superadministrador.\\
      \hline
      \textbf{Postcondiciones} & El administrador editará los datos de unos de los grupos existentes en el sistema.\\
      \hline
      \textbf{Escenario principal} & \smallskip 1. El sistema muestra la lista de grupos actuales.\\
      & 2. El administrador hace click en la opción de editar de la casilla del grupo deseado.\\
      & 3. El sistema muestra una ventana editable con los datos actuales del grupo.\\
      & 4. El administrador edita los datos deseados.\\
      & 5. El sistema valida los datos y registra los cambios, quedando reflejados en la lista de grupos.\\
      & \\
      \hline
      \textbf{Escenarios alternativos} & \smallskip 5.a. Los datos introducidos no son válidos:\\
      & \hspace{0.3cm} 5.a.1. El sistema muestra el mensaje de error correspondiente.\\
      & \hspace{0.3cm} 5.a.2. Vuelve al paso 4.\\
      & \\
      \hline
    \end{tabularx}
    \caption{UC-34: Gestión de servicios: Editar grupo}
  \end{center}
\end{table}


\begin{table}[H]
  \begin{center}
    \begin{tabularx}{16.4cm}{|l|X|}
      \hline
      \textbf{UC-35} & \textbf{Alta de rango de cita}\\
      \hline
      \textbf{Descripción} & Añadir un rango de horario dentro del cual estará disponible pedir citas para un determinado servicio. A partir de este momento, los usuarios podrán pedir hora para este servicio.\\
      \hline
      \textbf{Actores} & Administrador o superadministrador.\\
      \hline
      \textbf{Precondiciones} & Debe ser un usuario previamente identificado como administrador o superadministrador.\\
      \hline
      \textbf{Postcondiciones} & Se añadirá al sistema un nuevo rango para pedir citas del servicio seleccionado.\\
      \hline
      \textbf{Escenario principal} & \smallskip 1. El sistema muestra la lista de rangos de cita actuales.\\
      & 2. El administrador hace click en la opción de añadir un nuevo rango.\\
      & 3. El sistema muestra una ventana con los datos necesarios para la creación del rango.\\
      & 4. El administrador introduce los datos necesarios.\\
      & 5. El sistema valida los datos y registra el nuevo rango para pedir citas, quedando reflejado en la lista.\\
      & \\
      \hline
      \textbf{Escenarios alternativos} & \smallskip 5.a. Los datos introducidos no son válidos:\\
      & \hspace{0.3cm} 5.a.1. El sistema muestra el mensaje de error correspondiente.\\
      & \hspace{0.3cm} 5.a.2. Vuelve al paso 4.\\
      & \\
      \hline
    \end{tabularx}
    \caption{UC-35: Gestión de servicios: Alta de rango de cita}
  \end{center}
\end{table}


\begin{table}[H]
  \begin{center}
    \begin{tabularx}{16.4cm}{|l|X|}
      \hline
      \textbf{UC-36} & \textbf{Suspender/activar rango de cita}\\
      \hline
      \textbf{Descripción} & Suspender o activar uno de los rangos de cita de un determinado servicio del sistema.\\
      \hline
      \textbf{Actores} & Administrador o superadministrador.\\
      \hline
      \textbf{Precondiciones} & Debe ser un usuario previamente identificado como administrador o superadministrador.\\
      \hline
      \textbf{Postcondiciones} & El administrador activará o suspenderá unos de los rangos de cita existentes en el sistema.\\
      \hline
      \textbf{Escenario principal} & \smallskip 1. El sistema muestra la lista de rangos de cita actuales.\\
      & 2. El administrador hace click en la opción deseada (suspender/activar) de la casilla del rango deseado.\\
      & 3. El sistema suspende/activa el rango, quedando este inhabilitado/habilitado para que se pidan citas en el horario que comprende.\\
      & \\
      \hline
      \textbf{Escenarios alternativos} & \smallskip \\
      & \\
      \hline
    \end{tabularx}
    \caption{UC-36: Gestión de servicios: Suspender/activar rango de cita}
  \end{center}
\end{table}


\begin{table}[H]
  \begin{center}
    \begin{tabularx}{16.4cm}{|l|X|}
      \hline
      \textbf{UC-37} & \textbf{Editar rango de cita}\\
      \hline
      \textbf{Descripción} & Editar uno de los rangos de cita del sistema.\\
      \hline
      \textbf{Actores} & Administrador o superadministrador.\\
      \hline
      \textbf{Precondiciones} & Debe ser un usuario previamente identificado como administrador o superadministrador.\\
      \hline
      \textbf{Postcondiciones} & El administrador editará los datos de unos de los rangos existentes en el sistema.\\
      \hline
      \textbf{Escenario principal} & \smallskip 1. El sistema muestra la lista de rangos de cita actuales.\\
      & 2. El administrador hace click en la opción de editar de la casilla del rango deseado.\\
      & 3. El sistema muestra una ventana editable con los datos actuales del rango.\\
      & 4. El administrador edita los datos deseados.\\
      & 5. El sistema valida los datos y registra los cambios, quedando reflejados en la lista de rangos.\\
      & \\
      \hline
      \textbf{Escenarios alternativos} & \smallskip 5.a. Los datos introducidos no son válidos:\\
      & \hspace{0.3cm} 5.a.1. El sistema muestra el mensaje de error correspondiente.\\
      & \hspace{0.3cm} 5.a.2. Vuelve al paso 4.\\
      & \\
      \hline
    \end{tabularx}
    \caption{UC-37: Gestión de servicios: Editar rango de cita}
  \end{center}
\end{table}



\subsection{Actores} 
En este apartado se describirán los diferentes roles que juegan los usuarios que interactúan con el sistema. Los actores pueden ser roles de personas fí­sicas, sistemas externos o incluso el tiempo (eventos temporales).

\section{Modelo de Comportamiento}
A partir de los casos de uso anteriores, se crea el modelo de comportamiento. Para ello, se realizarán los diagramas de secuencia del sistema, donde se identificarán las operaciones o servicios del sistema. Luego, se detallará el contrato de las operaciones identificadas.

\section{Modelo de Interfaz de Usuario}
En esta sección se deberá incluir un prototipo de baja fidelidad o mockup de la interfaz de usuario del sistema. Además, es preciso elaborar un diagrama de navegación, reflejando la secuencia de pantallas a las que tienen acceso los diferentes roles de usuario y la conexión entre estas.

\chapter{Diseño del Sistema}
% !TEX encoding = UTF-8 Unicode
% ------------------------------------------------------------------------------
% Este fichero es parte de la plantilla LaTeX para la realización de Proyectos
% Final de Grado, protegido bajo los términos de la licencia GFDL.
% Para más información, la licencia completa viene incluida en el
% fichero fdl-1.3.tex

% Copyright (C) 2012 SPI-FM. Universidad de Cádiz
% ------------------------------------------------------------------------------

A lo largo de este capítulo se detallará la arquitectura general del sistema de información, el diseño fí­sico de datos, el diseño detallado de componentes software y el diseño de la interfaz de usuario:

\section{Arquitectura del Sistema}

En esta sección se define la arquitectura general del sistema de información, especificando la infraestructura tecnológica necesaria para dar soporte al software y la estructura de los componentes que lo forman.


\subsection{Arquitectura Fí­sica} \label{sec:arquitectura-fisica}

El desarrollo de este proyecto no precisa de ningún elemento hardware adicional al equipo de trabajo del alumno. Se ha utilizado un portátil MacBook Pro de 15 pulgadas, con procesador Intel Core i7 de 2.2 GHz y 16GB de memoria RAM DDR3. Para la realización de pruebas, se usará tanto este equipo como otro portátil del propio alumno, donde se instalará todo el software requerido para comprobar que la instalación y ejecución de la aplicación responde adecuadamente en un equipo diferente. En este caso será un portátil Acer Aspire 5732z con procesador Dual Core, 4GB de memoria RAM y disco duro SSD. \\

Respecto al software, el MacBook trabaja bajo el sistema operativo macOS Sierra. Todo el proyecto se ha desarrollado utilizando el IDE NetBeans 8.0.2 y usando el servidor de aplicaciones GlassFish en un entorno local. Para la documentación, se ha utilizado TeXShop, como herramienta de edición para \LaTeX. Para las pruebas, el portátil a utilizar correrá bajo Windows 7, usando el mismo IDE y servidor de aplicaciones. \\

Respecto al entorno de producción, la aplicación web se alojará en un servidor que se contratará para tal fin. Por lo tanto, solo se requerirá acceso al servidor para la instalación de, en este caso, el servidor de aplicaciones Wildfly (JBoss) -habiendo sido probado previamente en entorno local de desarrollo-, siendo este similar a GlassFish, por lo que la aplicación apenas requerirá cambio alguno y aportará algo más de robustez y calidad. Los usuarios del sistema podrán hacer uso del mismo utilizando cualquier dispositivo con acceso a internet a través de un navegador web, como sus propios móviles, tablets o PCs. \\

En el apartado \ref{sec:entorno-construcción} se describirá detalladamente todo el software, lenguaje, frameworks, etc. utilizados para el desarrollo del sistema.


\subsection{Arquitectura Lógica} \label{sec:arquitectura-logica}

Para el desarrollo de esta aplicación web se ha utilizado una arquitectura de 3 capas, basada en el patrón de diseño Modelo-Vista-Controlador (MVC), donde la primera capa correspondería a la capa de usuario, la segunda la de negocio y por último la capa de datos.

\vspace{10mm}

\begin{figure}[H]
\centering
  \includegraphics[scale=.55]{img/arquitectura-tres-capas.jpg}
  \caption{Representación de Arquitectura de 3 Capas}
  \label{fig:arquitectura-tres-capas}
\end{figure}

\vspace{10mm}

La programación por capas es un modelo de desarrollo software en el que el objetivo primordial es la separación (desacomplamiento) de las partes que componen un sistema software o también una arquitectura cliente-servidor: lógica de negocios, capa de presentación y capa de datos. De esta forma, por ejemplo, es sencillo y mantenible crear diferentes interfaces sobre un mismos sistema sin requerirse cambio alguno en la capa de datos o lógica.\\

La ventaja principal de este estilo es que el desarrollo se puede llevar a cabo en varios niveles y, en caso de que sobrevenga algún cambio, solo afectará al nivel requerido sin tener que revisar entre el código fuente de otros módulos (\cite{bib:wikipedia-programacion-por-capas}).

\paragraph*{Capa de presentación (frontend)}

Este grupo de artefactos software conforman la capa de presentación del sistema, incluyendo tanto los componentes de la vista como los elementos de control de la misma. \\

Es la capa que ve el usuario, denominada también \textit{capa de usuario}. Presenta el sistema al usuario, le comunica la información y captura la información del usuario en un mínimo de proceso (realiza un filtrado previo para comprobar que no hay errores de formato). También es conocida como interfaz gráfica y debe tener la característica de ser amigable (entendible y fácil de usar) para el usuario. Esta capa se comunica únicamente con la capa de negocio.

\paragraph*{Capa de negocio}

Esta capa recibe las peticiones del usuario y se envían las respuestas tras el proceso. Se denomina capa de negocio (o de lógica del negocio) porque es aquí donde se establecen todas las reglas que deben cumplirse. Esta capa se comunica con la capa de presentación, para recibir las solicitudes y presentar los resultados, y con la capa de datos, para solicitar al gestor de base de datos almacenar o recuperar datos de él. 

\paragraph*{Capa de persistencia}

Este grupo de artefactos software conforman la capa de integración del sistema, incluyendo las clases de abstracción para el acceso a datos.\\

Es donde residen los datos y es la encargada de acceder a los mismos. Está formada por un gestor de base de datos que realiza todo el almacenamiento de datos, recibe solicitudes de almacenamiento o recuperación de información desde la capa de negocio.\\

Es común que a la capa de negocio y de datos de los sistemas web se denomine conjuntamente como backend de la aplicación.\\

Como se ha comentado anteriormente, el patrón de diseño usado para el desarrollo del proyecto ha sido Modelo-Vista-Controlador, el cual separa los datos y la lógica de negocio de una aplicación de la interfaz de usuario y el módulo encargado de gestionar los eventos y las comunicaciones. Para ello MVC propone la construcción de tres componentes distintos que son el modelo, la vista y el controlador, es decir, por un lado define componentes para la representación de la información, y por otro lado para la interacción del usuario.​ Este patrón de arquitectura de software se basa en las ideas de reutilización de código y la separación de conceptos, características que buscan facilitar la tarea de desarrollo de aplicaciones y su posterior mantenimiento (\cite{bib:wikipedia-modelo-vista-controlador}).

\vspace{10mm}

\begin{figure}[H]
\centering
  \includegraphics[scale=.55]{img/proceso-MVC.jpg}
  \caption{Proceso del Patrón MVC}
  \label{fig:proceso-MVC}
\end{figure}

\vspace{10mm}

\paragraph*{Modelo}

Es la representación de la información con la cual el sistema opera, por lo tanto gestiona todos los accesos a dicha información, tanto consultas como actualizaciones, implementando también los privilegios de acceso que se hayan descrito en las especificaciones de la aplicación (lógica de negocio). Envía a la \textit{vista} aquella parte de la información que en cada momento se le solicita para que sea mostrada al usuario. Las peticiones de acceso o manipulación de información llegan al \textit{modelo} a través del \textit{controlador}.

\paragraph*{Controlador}

Responde a eventos (acciones del usuario) e invoca peticiones al \textit{modelo} cuando se hace alguna solicitud sobre la información (por ejemplo, editar un documento o un registro en la base de datos). También puede enviar comandos a su \textit{vista} asociada si se solicita un cambio en la forma en que se presenta el \textit{modelo} (por ejemplo, desplazamiento o scroll por un documento o por los diferentes registros de una base de datos), por tanto se podría decir que el \textit{controlador} hace de intermediario entre la \textit{vista}  y el \textit{modelo}.

\paragraph*{Vista}

Presenta el \textit{modelo} (información y lógica de negocio) en un formato adecuado para interactuar (la interfaz de usuario), por tanto requiere de dicho \textit{modelo} la información que debe representar como salida.\\

Por tanto, aunque la arquitectura de 3 capas o niveles y el patrón MVC presenten sus similitudes y diferencias, cada uno tiene su función y son compatibles entre sí, de ahí el uso de ambos en el presente proyecto. A continuación, se presenta una gráfica comparativa de ambos modelos. 

\vspace{10mm}

\begin{figure}[H]
\centering
  \includegraphics[scale=.50]{img/MVC-vs-3-capas.jpg}
  \caption{Comparativa entre modelo MVC y arquitectura de 3 capas}
  \label{fig:MVC-vs-3-capas}
\end{figure}

\vspace{10mm}


\section{Diseño Fí­sico de Datos}

Habiendo realizado previamente el modelo de conceptual de clases, detallado en la sección \ref{sec:modelo-conceptual}, se puede tener una idea de la estructura física que tendrá los datos en el sistema de gestión de base de datos (SGBD) a utilizar, en este caso PostgreSQL, teniendo en cuenta que aparecerán nuevas tablas en la BD provenientes de las relaciones entre las clases. Pero, por supuesto, hay que tener en cuenta que el acceso a los mismos se realice de una forma eficaz e independiente al resto de la implementación.\\ 

Y es por ello por lo que la arquitectura lógica del sistema se divide en 3 capas bien diferenciadas. La tercera de las capas contendrá el DAO (\textit{Data Access Object, Objeto de Acceso a Datos}), encargado del acceso a los datos físicos y única capa que realizará cambios en los mismos. Esto permite que si aflora la necesidad de cambios en la estructura de nuestros datos, o incluso cambiar de SGBD, las demás capas queden totalmente al margen de estos cambios, siendo un trámite independiente sin afectar -dependiendo del cambio, claro está- a la lógica de negocio o la interfaz de usuario. 


\section{Diseño de la Interfaz de Usuario} 

A continuación se muestra un prototipo de la interfaz de usuario del sistema para PC. Se mostrarán las páginas principales de la aplicación web de manera generalizada: pantalla de registro de usuario, inicio de sesión, pantalla generalizada de la aplicación una vez iniciada la sesión y página con prototipo de tabla de datos (para listado de usuarios, servicios, clases, etc.). 


\vspace{10mm}

\begin{figure}
\centering
  \includegraphics[scale=.40]{img/interfaz/inicio-sesion.jpg}
  \caption{Interfaz de usuario: Inicio de sesión}
  \label{fig:interfaz-inicio-sesion}
\end{figure}

\begin{figure}
\centering
  \includegraphics[scale=.40]{img/interfaz/registro.jpg}
  \caption{Interfaz de usuario: Registro}
  \label{fig:interfaz-registro}
\end{figure}

\begin{figure}
\centering
  \includegraphics[scale=.40]{img/interfaz/pantalla-principal.jpg}
  \caption{Interfaz de usuario: Pantalla general}
  \label{fig:interfaz-pantalla-principal}
\end{figure}

\begin{figure}
\centering
  \includegraphics[scale=.40]{img/interfaz/cuadro-general.jpg}
  \caption{Interfaz de usuario: Pantalla con tabla de datos}
  \label{fig:interfaz-cuadro-general}
\end{figure}


Asimismo, se ha realizado el diseño de las pantallas para dispositivos de menor tamaño, al ser un diseño adaptativo dependiendo del mismo. A continuación se podrán visualizar los mockups realizados para la interfaz de dispositivos móviles. En este caso, las pantalla de inicio de sesión, pantalla general de usuario y menú desplegado.


\begin{figure}[H]
\centering
  \includegraphics[scale=.50]{img/interfaz/inicio-sesion-movil.jpg}
  \caption{Interfaz de usuario para dispositivos móviles: Inicio de sesión}
  \label{fig:interfaz-inicio-sesion-movil}
\end{figure}

\begin{figure}[H]
\centering
  \includegraphics[scale=.50]{img/interfaz/pantalla-principal-movil.jpg}
  \caption{Interfaz de usuario para dispositivos móviles: Pantalla general}
  \label{fig:interfaz-pantalla-principal-movil}
\end{figure}

\begin{figure}[H]
\centering
  \includegraphics[scale=.50]{img/interfaz/menu-movil.jpg}
  \caption{Interfaz de usuario para dispositivos móviles: Menú desplegable}
  \label{fig:interfaz-menu-movil}
\end{figure}


Además, la figura \ref{fig:interfaz-navegacion} muestra el diagrama de navegación entre pantallas. Observamos que es una navegación sencilla; la página que se mostraría al acceder a la aplicación sería la de inicio de sesión. Si se trata de un usuario registrado, podrá acceder directamente a la página principal de la aplicación (\textit{Home}) a través del usuario y contraseña. En caso contrario, habría que navegar a la página de registro para que, una vez registrado, pueda acceder al sistema llegando a la página principal mencionada. Desde esta página de inicio (\textit{Home}) se podrá navegar, a través del menú y/o enlaces disponibles, hasta las distintas vistas de la interfaz. En todo momento será posible cerrar la sesión del usuario, volviendo a la página de inicio de sesión, o cambiar el idioma de la interfaz mediante la opción destinada a ello en la parte superior derecha de la pantalla. \\ 

En el diagrama observamos que se navega hacia la vista de una tabla de datos. Este es un ejemplo de tantas vistas como hay en el sistema. Algunos ejemplos de tablas de datos pueden ser las página donde se listan los servicios, clases, citas, usuarios (para administradores) o reservas del usuario. Existen muchas más páginas en el sistema para navegar, como las páginas de perfil, cambio de contraseña, bandeja de entrada, calendario, creación/edición de servicios/clases/citas/usuarios para administradores, etc. \\

Esta navegación ocurrirá de la misma manera en todo tipo de dispositivos, donde el único cambio sería el diseño de la pantalla, como hemos observado en los prototipos para PC y móvil.


\begin{figure}[H]
\centering
  \includegraphics[scale=.50]{img/interfaz/navegacion.jpg}
  \caption{Interfaz de usuario: Diagrama de navegación}
  \label{fig:interfaz-navegacion}
\end{figure}



\chapter{Construcción del Sistema}
% !TEX encoding = UTF-8 Unicode
% ------------------------------------------------------------------------------
% Este fichero es parte de la plantilla LaTeX para la realización de Proyectos
% Final de Grado, protegido bajo los términos de la licencia GFDL.
% Para más información, la licencia completa viene incluida en el
% fichero fdl-1.3.tex

% Copyright (C) 2012 SPI-FM. Universidad de Cádiz
% ------------------------------------------------------------------------------

Este capí­tulo trata sobre todos los aspectos relacionados con la implementación del sistema en código, haciendo uso de un determinado entorno tecnológico.

\section{Entorno de Construcción}\label{sec:entorno-construcción}

Como se ha especificado en la sección \ref{sec:arquitectura-fisica}, el desarrollo de este proyecto ha sido realizado haciendo uso del equipo del propio alumno, sin necesidad de alguna herramienta hardware extra. Para ello, se ha hecho uso de un marco tecnológico específico que se detallará a continuación: 

\paragraph*{Hardware}

Los elementos del hardware utilizados no son relevantes para el desarrollo del sistema, ya que no se requiere nada fuera de lo común en un equipo de trabajo convencional. En este caso, se ha utilizado un portátil MacBook Pro de 15 pulgadas, con procesador Intel Core i7 de 2.2 GHz y memoria RAM de 16GB 1333 MHz DDR3.

\paragraph*{IDE (Entorno de Desarrollo Integrado)}

NetBeans es un IDE libre y gratuito pensado especialmente en desarrollo de software bajo el uso del lenguaje de programación Java.

\paragraph*{Lenguaje de Programación} 

Para la realización de la aplicación web se ha utilizado el lenguaje de programación \textbf{Java}, en concreto la plataforma Java EE (Enterprise Edition), con la ayuda de varios frameworks para diferentes cometidos, como son JSF, PrimeFaces, EJB y JPA, que se describirán a continuación.

\paragraph*{Frameworks}

\begin{itemize}
\item \textbf{JSF (JavaServer Faces):} framework MVC que proporciona un conjunto de componentes en forma de etiquetas definidas en páginas XHTML mediante el framework Facelets. Se utiliza para aplicaciones Java basadas en web simplificando el desarrollo de interfaces de usuario. 
\item \textbf{Facelets:} framework basado que permite definir la estructura general de las páginas (su layout) mediante plantillas. Facelets se adapta perfectamente al enfoque de JSF y se incorpora a la especificación desde la revisión 2.1. Anteriormente, las páginas JSF se definían utilizando etiquetas específicas de JSP, lo que generaba cierta confusión porque se trata de enfoques alternativos para un mismo problema. La sustitución de JSP por Facelets como lenguaje básico para definir la disposición de las páginas permite separar perfectamente las responsabilidades de cada parte del framework. La estructura de la página se define utilizando las etiquetas Facelets y los componentes específicos que deben presentar los datos de la aplicación utilizando etiquetas JSF. Para más información sobre JSF y/o Facelets véase el enlace \ref{bio:introduccion-jsf} de la biografía.
\item \textbf{PrimeFaces:} este framework es una extensión de JSF de código abierto que cuenta con un conjunto de componentes enriquecidos para facilitar la creación de interfaces de usuario.
\item \textbf{EJB (Enterprise JavaBeans):} plataforma para construir aplicaciones empresariales portables, reusables y escalables, utilizando el lenguaje de programación java. EJB permite a los desarrolladores de aplicaciones enfocarse en construir la lógica de negocio sin la necesidad de gastar tiempo en la construcción de código de infraestructura. 
\item \textbf{JPA (Java Persistence API:} La persistencia dentro de EJB es administrada por JPA. JPA permite persistir automáticamente los objetos Java utilizando una técnica denominada object-relational mapping (ORM). ORM es esencialmente el proceso de mapear la información contenida en los objetos Java hacia las tablas de base de datos utilizando una configuración.
JPA define un estándar para:
\begin{itemize}
\item La creación de configuración metadata del ORM para mapear entidades hacia tablas relacionales.
\item La EntityManager API, una API estándar para realizar las operaciones CRUD (create, read, update y delete) de las entidades.
\item El lenguaje Java Persistence Query Language (JPQL), para realizar búsquedas y obtener información persistida de la aplicación.
\end {itemize}
\end {itemize}

En la siguiente figura podemos ver una representación de la integración de los frameworks descritos en la arquitectura de 3 capas vista en la sección \ref{sec:arquitectura-logica}.

\vspace{10mm}

\begin{figure}[H]
\centering
  \includegraphics[scale=.75]{img/arquitectura-jee.jpg}
  \caption{Arquitectura de 3 Capas con Frameworks}
  \label{fig:arquitectura-jee}
\end{figure}

\vspace{10mm}

\paragraph*{SGBD}

\paragraph*{Control de Versiones}



\subsection{Entorno para la Web Pública}


En esta sección se debe indicar el marco tecnológico utilizado para la construcción del sistema: entorno de desarrollo (IDE), lenguaje de programación, herramientas de ayuda a la construcción y despliegue, control de versiones, repositorio de componentes, integración contí­nua, etc.



\section{Código Fuente}
Organización del código fuente, describiendo la utilidad de los diferentes ficheros y su distribución en paquetes o directorios. Asimismo, se incluirá algún extracto significativo de código fuente que sea de interés para ilustrar algún algoritmo o funcionalidad especí­fica del sistema.

\section{Scripts de Base de Datos}
Organización del código fuente, describiendo la utilidad de los diferentes ficheros y su distribución en paquetes o directorios. Asimismo, se incluirá el script de algún disparador o un procedimiento almacenado, que sea de interés para ilustrar algún aspecto concreto de la gestión de la base de datos.


\chapter{Pruebas del Sistema}
% !TEX encoding = UTF-8 Unicode
% ------------------------------------------------------------------------------
% Este fichero es parte de la plantilla LaTeX para la realización de Proyectos
% Final de Grado, protegido bajo los términos de la licencia GFDL.
% Para más información, la licencia completa viene incluida en el
% fichero fdl-1.3.tex

% Copyright (C) 2012 SPI-FM. Universidad de Cádiz
% ------------------------------------------------------------------------------

En este capí­tulo se presenta el plan de pruebas del sistema de información, incluyendo los diferentes tipos de pruebas que se han llevado a cabo.


\section{Entorno de Pruebas}

Como se mencionó en la sección \textit{Arquitectura Física} \ref{sec:arquitectura-fisica}, para la realización de las pruebas de este PFC se utilizarán dos ordenadores portátiles del propio alumno: 

\begin{itemize}
\item El primero, un MacBook Pro de 15 pulgadas, con procesador Intel Core i7 de 2.2 GHz y 16GB de memoria RAM DDR3, usando, por tanto, Mac OS y todos los elementos software nombrados en la sección \textit{Entorno de Construcción} \ref{sec:entorno-construcción}. 
\item El segundo equipo, un Acer Aspire 5732z con procesador Dual Core, 4GB de memoria RAM y disco duro SSD y Windows 7, con los mismos programas y frameworks, los necesarios para la instalación y ejecución del sistema.
\end{itemize}


\section{Roles}

La mayor parte de las pruebas serán realizadas por el alumno en sus equipos, utilizando perfiles de los diferentes tipos de usuarios posibles en el sistema. Una vez estas hayan sido llevadas a cabo, se procederá a evaluar el sistema con los dos miembros que componen la empresa, para verificar requisitos y usabilidad.


\section{Niveles de Pruebas}

Las pruebas realizadas en el sistema han sido en su totalidad de forma manual. \\

Durante el desarrollo del mismo se han ido realizando pruebas unitarias para corroborar el funcionamiento de las funciones que se iban desarrollando, además de realizar pruebas de integración cuando se finalizaban módulos completos, como por ejemplo la creación y gestión de citas y su integración en el calendario de actividades, probándose la funcionalidad del entorno de citas y su interacción.\\ 

Una vez finalizado cada ciclo de desarrollo, con los objetivos propuestos en cada uno, se realizan también pruebas de sistema, comprobando que las nuevas funcionalidades no han afectado a la funcionalidad del resto del sistema y el funcionamiento del mismo es el adecuado.\\ 

En la finalización del sistema completo, se volverán a realizar todas las pruebas mencionadas siguiendo los mismos procesos, como se explica a continuación.


\subsection{Pruebas Unitarias}

Se realizan pruebas unitarias para cada funcionalidad del sistema, poniendo especial atención a los detalles y que cada una de ellas tenga el funcionamiento esperado, como por ejemplo la función de todos los elementos de cada formulario, que se muestren correctamente los mensajes de error o de información, los datos introducidos se comprueban correctamente, etc. \\ 

El sistema pasa con éxito las pruebas y se procede a las pruebas de integración.


\subsection{Pruebas de Integración}

Una vez comprobado que el sistema pasa correctamente las pruebas unitarias, se procede a las de integración. \\

Se comprueba que los conjuntos de funcionalidades o módulos funcionan correctamente e interactúan entre sí de forma esperada. Por ejemplo, se comprobará que todos los datos introducidos por el usuario se guardan correctamente en la BD pasando por las capas correspondientes o que el número de plazas disponibles en una clase se reduce cuando un usuario reserva.\\

De este modo, se comprueba que la funcionalidad de los módulos es correcta.


\subsection{Pruebas de Sistema}

Una vez el sistema parece funcionar de una forma adecuada y sin errores, se procede a realizar pruebas funcionales y no funcionales, de acuerdo a los requisitos establecidos en el catálogo de requistos \ref{sec:catalogo-requisitos}. 


\subsubsection{Pruebas Funcionales}

Con estas pruebas se analiza el buen funcionamiento de la implementación de los flujos normales y alternativos de los distintos casos de uso del sistema. Así, iremos probando cada requisito funcional: 

\begin{itemize}
\item Se comprueba que la opción de \textbf{selección} de idioma está presente y realiza el cambio correctamente.
\item Un nuevo usuario puede realizar su \textbf{registro} obteniendo acceso al sistema.
\item El \textbf{acceso al sistema} para usuarios registrados se hace como se espera.
\item La \textbf{sesión queda cerrada} de forma correcta, siendo obligatorio volver a identificarse para acceder al sistema.
\item En la página de inicio se observa las \textbf{notificaciones} del usuario.
\item Los usuarios tienen acceso a la bandeja de su \textbf{correo}.
\item Aquí, es posible navegar a la edición de correos y \textbf{enviar uno nuevo} adecuadamente.
\item La opción de \textbf{restablecer contraseña} se prueba y se recibe el correo correctamente, pudiéndose completar la acción.
\item Asimismo, una vez identificado, el usuario puede \textbf{cambiar su contraseña} de forma adecuada.
\item También sus \textbf{datos del perfil}, quedando los cambios reflejados en el sistema.
\item Las clases disponibles son mostradas de forma correcta y el usuario puede \textbf{realizar reservas} en las mismas.
\item Así como \textbf{cancelar dichas reservas}.
\item En cuanto a \textbf{citas}, vemos que el sistema muestra también un correcto comportamiento y el usuario puede \textbf{solicitarlas}.
\item El administrador, por su parte, \textbf{responde a las solicitudes de cita} de forma adecuada, llegándole la notificación al usuario.
\item El propio usuario puede \textbf{cancelarla}, siendo el administrador quien recibe la solicitud en caso de estar aceptada.
\item Todas las citas y clases, tanto reservadas como disponibles y pasadas, pueden consultarse en el \textbf{calendario de actividades} diponible.
\item Otra opción sería \textbf{consultar las reservas} del usuario en la página correspondiente, donde se observa que se muestran correctamente.
\item Se comprueba que, tanto el perfil de usuario como los de administración, pueden ver todas las acciones realizadas por él mismo o por otros usuarios (solo administradores), mediante la página de \textbf{auditorías}.
\item Las opciones de administración de usuarios también son testadas, siendo posible \textbf{activar/suspender usuarios, ver y editar sus perfiles, así como activar/suspender y ver los perfiles de otros administradores.}
\item El superadministrador, además, puede editar el perfil de los administradores sin ningún tipo de error.
\item Respecto a al gestión de servicios, los dos roles de administración pueden \textbf{dar de alta, activar, suspender y editar servicios.}
\item De la misma forma, la \textbf{gestión de clases} se realiza correctamente, estando disponibles las mismas opciones.
\item El \textbf{alta, activación y suspensión de cita} también se ejecutan con éxito.
\item Por último, se comprueba la gestión de \textbf{archivos}, donde la \textbf{creación y edición} de los mismos parece correcta.
\item Tanto el administrador como los usuarios a los que van dirigido son capaces de realizar su \textbf{descarga del archivo}.
\end{itemize}

\subsubsection{Pruebas No Funcionales}

Respecto a las pruebas de los requisitos no funcionales identificados en la subsección \ref{subsec:requisitos-no-funcionales} los resultados han sido:

\begin{itemize}
\item \textbf{Disponibilidad:} Este requisito dependerá del servidor donde se aloje el producto final. Todavía no se han realizado pruebas de producción, solo de desarrollo. Una vez se contrate un servidor se realizarán las pruebas pertinentes. En principio, un servidor debe proporcionar el servicio adecuado para cumplir este requisito, siendo la aplicación alojada en él accesible 24 horas.
\item \textbf{Fiabilidad:} Por una parte, se realizan pruebas de testeo para asegurarnos que el sistema no posee ningún error. Por otro, se utilizan sesiones Java para los usuarios y se encriptan las contraseñas para ofrecer más seguridad a la aplicación, quedando guardadas en base de datos de esta manera, con el objetivo de que este requisito se cumpla correctamente.
\item \textbf{Internacionalización:} Se comprueba que la opción de traducción se realiza correctamente, eligiendo el idioma deseado en el desplegable que el sistema muestra en todo momento. 
\item \textbf{Usabilidad:} Se ha desarrollado una interfaz intuitiva de fácil acceso y uso, así como adaptada a distintos dispositivos, cumpliendo con este requisito. Aunque se recomiendo su uso en pantallas medianas o grandes, como tablets u ordenadores, debido al tamaño de algunas tablas de datos y mayor facilidad de uso por el espaciado.
\item \textbf{Mantenibilidad:} Esta prueba se realizará a lo largo de la vida del sistema. En principio, hará falta poco mantenimiento y la opción de escalabilidad, ya sea para su uso con otras empresas o para añadir nuevas funcionalidades, ha sido tenida en cuenta en el desarrollo del proyecto para facilitarlo.
\end{itemize}

\subsection{Pruebas de Aceptación}

Una vez todas las pruebas han sido realizadas, se realizan a nivel general con los clientes finales, tanto administradores, que han ido interactuando con el sistema a lo largo de su desarrollo, como usuarios voluntarios del centro. De esta manera, se busca obtener un feedback, tanto en la facilidad de uso como sensaciones de los usuarios. \\ 

Estas pruebas de aceptación resultan exitosas. Si bien es cierto que han sido realizadas con una pequeña muestra en entorno de desarrollo. Las mismas se realizarán en entorno de producción con una muestra de testeadores mayor y por un periodo algo más prolongado, antes de su uso definitivo.


% EPILOGO
\part{Epí­logo}
\null\vfill
\noindent En esta última parte quedarán recogidas las conclusiones y los manuales necesarios para el manejo de la aplicación resultado del desarrollo. Si se ha realizado algún tipo de evaluación de la solución proporcionada, más allá de las pruebas del sistema, tambií©n deberá venir recogida en un capí­tulo separado dentro de esta parte. Pueden consultarse diversos tipos de evaluaciones sobre sistemas de información en \cite{hevner2004}: casos de estudio, análisis estático, análisis dinámico, simulación, experimento controlado, etc.
\vfill

\chapter{Manual de implantación y explotación}
% !TEX encoding = UTF-8 Unicode
% ------------------------------------------------------------------------------
% Este fichero es parte de la plantilla LaTeX para la realización de Proyectos
% Final de Grado, protegido bajo los términos de la licencia GFDL.
% Para más información, la licencia completa viene incluida en el
% fichero fdl-1.3.tex

% Copyright (C) 2012 SPI-FM. Universidad de Cádiz
% ------------------------------------------------------------------------------

Las instrucciones de instalación y explotación del sistema se detallan a continuación.

\section{Introducción}

El presente software está destinado a la gestión de un centro de mejora de la salud y el rendimiento, en concreto, ha sido una personalización para el centro \textit{CoreSport}, en Chiclana de la Frontera (Cádiz). \\

Cualquier centro similar que precise de un software de gestión puede hacer uso del mismo con o sin modificaciones, bajo la licencia GNU GPL (Licencia Pública General de GNU) en su versión 3 o superior (\ref{GNU-license}). 


\section{Requisitos previos}

Ha de diferenciarse dos tipos de usos o instalaciones del sistema: 

\begin{itemize} 
\item Sin modificaciones: Si se pretende hacer uso del software tal y como se entrega, sin necesidad de un desarrollo previo para su modificación o ampliación, habría que hacer uso de un servidor para la instalación de \textit{GlassFish} o \textit{WildFly (JBoss)} (recomendado), o cualquier otro servidor de aplicaciones compatible para, posteriormente, realizar la instalación del software. Para ello, bastaría con el archivo \textit{Booking.ear} contenido en el directorio \textit{dir}, el cual se puede cargar directamente en el servidor, junto con el backup de la base de datos con los datos básicos para poder iniciar el sistema. Se añadirá un manual de instalación para entornos de producción en un futuro cercano. 
\item También sería posible, una vez el sistema se encuentre en producción, utilizar el mismo servidor para varias empresas, lo que facilitaría la instalación -ya estaría hecha- y reduciría el precio del servicio al ser compartido. En este caso, solo habría que añadir la nueva organización y sus elementos para la interfaz (logo, icono, estilo...). 
\item Instalación para desarrollo: En caso que se requiera una instalación local para desarrollo, la instalación sería diferente. Para ello, cualquier equipo básico con un rendimiento aceptable sería suficiente para la instalación. Respecto al software, haría falta, aparte del código del propio proyecto, la instalación de \textit{JDK, NetBeans, PostgreSQL, pgAdmin} y un servidor de aplicaciones, como \textit{GlassFish} (recomendado por su \textit{hot deployment}) o \textit{WildFly (JBoss)}, junto a los archivos \textit{jar} de \textit{Primefaces, PostgreSQL, Commons IO} y \textit{JavaEE Endorsed API}, librerías de las que se beneficiará el PFC.
\end{itemize} 


\section{Inventario de componentes}

Al descargar una copia del código fuente del PFC, se obtiene lo siguiente: 

\begin{itemize}
\item \textbf{Código fuente del proyecto}: Código Java de la aplicación web.
\item Últimas versiones disponibles -en septiembre de 2017- de los \textbf{archivos .jar} necesarios, contenidos en el directorio \textit{lib: Primefaces, PostgreSQL, Commons IO} y \textit{JavaEE Endorsed API}, además de los drivers de \textit{PostgreSQL} para \textit{GlassFish} (\textit{postgresql-9.3-1102.jdbc41.jar}).
\item Copia de una \textbf{base de datos} básica, con una organización, un superadministrador, un administrador y un cliente. Además de un \textbf{archivo explicativo} donde se facilitan los datos de estos usuarios y contraseñas y se explica cómo crear estos datos de forma personalizada. 
\item \textbf{Memoria del proyecto}, tanto en PDF como en \LaTeX, para que pueda ser actualizada si se desea. 
\end{itemize}


\section{Procedimientos de instalación}

A continuación se detallarán todos los pasos necesarios para la instalación del sistema en un equipo para su desarrollo o para pruebas locales. \\

Primeramente se procederá a la descarga de todos los elementos software necesarios: 

\begin{itemize}
\item \textit{JDK (Java Development Kit)} \cite{JDK}, que es el software que proporciona las herramientas para el desarrollo de aplicaciones Java.
\item \textit{NetBeans}, IDE (Entorno de Desarrollo Integrado) para el desarrollo de aplicaciones. Pensado principalmente para aplicaciones Java.
\item \textit{GlassFish}, servidor de aplicaciones para el desarrollo del proyecto en local.
\item \textit{Oracle} también ofrece la posibilidad de descargar \textit{NetBeans} con \textit{JDK8} -e incluso con \textit{GlassFish}- para que sea más cómodo y rápido \cite{NetBeans-con-JDK}.
\item Código fuente del proyecto, disponible en el repositorio \textit{GitHub} destinado a ello \cite{github-booking}.
\item \textit{PostgreSQL} \cite{PostgreSQL}, sistema de gestión de bases de datos a utilizar.
\item \textit{pgAdmin} \cite{pgAdmin}, interfaz de usuario para la administración de bases de datos PostgreSQL.
\item Librerías \textit{.jar} indicadas, que se encuentran en el directorio \textit{lib}, aunque se debe tener en cuenta la existencia de nuevas versiones. Las librerías serían:

\begin{itemize}
\item \textit{Commons IO}
\item \textit{Primefaces}
\item \textit{JavaEE Endorsed API}
\end{itemize}
\end{itemize}

\item El archivo .jar perteneciente al driver de \textit{PostgreSQL} para \textit{GlassFish: postgresql-9.3-1102.jdbc41.jar}.

Una vez descargado todo lo necesario, se procederá a la instalación de cada uno de ellos, junto a la creación y configuración del proyecto:

\begin{enumerate}
\item Se instalará \textit{NetBeans} con \textit{JDK} -actualmente JDK8-, ya sea en la misma instalación o por separados. 

\item Asimismo, se procederá a la instalación de \textit{GlassFish} si no se ha realizado en conjunto con la anterior. Si se hace por separado, habría que añadir el servidor a \textit{NetBeans} a través de la pestaña \textit{Services}, haciendo clic derecho en \textit{Servers} y añadiendo el servidor, eligiendo \textit{GlassFish} y a continuación la ruta donde se encuentra la carpeta descargada.  

\item Seguidamente, se creará un nuevo proyecto en \textit{NetBeans}:

\begin{enumerate}
\item Se elegirá la categoría \textit{Java EE} y el tipo de proyecto \textit{Enterprise Application with Existing Sources}, para crear el proyecto a partir del código fuente descargado. 

\begin{figure}[H]
\centering
  \includegraphics[scale=.60]{img/instalacion/nuevo-proyecto.jpg}
  \caption{\textit{Creación de un nuevo proyecto en NetBeans}}
  \label{fig:nuevo-proyecto}
\end{figure}

\item En el siguiente paso de la creación del proyecto, se selecciona la ubicación del mismo. Podemos cambiar el nombre y ubicación del proyecto en este paso, así como indicarle a \textit{NetBeans} cuál es el directorio que vamos a utilizar para almacenar las librerías que se van a usar activando la casilla \textit{Use Dedicated Folder for Storing Libraries}, como vemos en la figura \ref{fig:nuevo-proyecto}.

\begin{figure}[H]
\centering
  \includegraphics[scale=.55]{img/instalacion/ubicacion-nuevo-proyecto.jpg}
  \caption{\textit{Creación del proyecto: Ubicación}}
  \label{fig:ubicacion-nuevo-proyecto}
\end{figure}

\item En el próximo paso elegiremos el servidor de aplicaciones a emplear, en este caso \texti{GlassFish}, y la versión Java EE, eligiendo en ambas opciones su versión más reciente. 

\item Por último en cuanto a la creación del proyecto se refiere, se observa que el módulo Web \texti{Booking-war} se ha añadido automáticamente. Añadimos también el módulo EJB haciendo clic en la opción para ello y eligiendo el directorio \texti{Booking-ejb}. Una vez añadido, seleccionamos que se trata del módulo EJB, como vemos en la imagen \ref{fig:modulos-proyecto}.

\begin{figure}[H]
\centering
  \includegraphics[scale=.55]{img/instalacion/modulos-proyecto.jpg}
  \caption{\textit{Definición de los módulos del proyecto}}
  \label{fig:modulos-proyecto}
\end{figure}

\end{enumerate}

\item Una vez el proyecto ha sido creado, definiremos dónde se encuentra nuestro módulo web a través del \textit{context root} del archivo \textit{application.xml} que encontraremos en el directorio \textit{Configuration Files} del proyecto. Cambiaremos la fila dedicada a ello, definiendo el \textit{context root} como sigue: \textit{<context-root/>}, de tal forma que le indicaremos al sistema que el módulo se encuentra en el directorio raíz, no haciendo falta ruta alguna para llegar a él.

\begin{figure}
\centering
  \includegraphics[scale=.65]{img/instalacion/context-root.jpg}
  \caption{\textit{Estableciendo el context root del archivo application.xml}}
  \label{fig:context-root}
\end{figure}

\item En este momento, se puede observar que existen diferentes advertencias de errores, las solucionaremos como sigue:

\begin{enumerate}
\item En el módulo \textit{Web} existen advertencias de errores, a través de iconos en los directorios donde estos aparecen; se trata de la falta de las librerías de las que el código hace uso. Por lo que seguidamente instalaremos estas dependencias del proyecto, es decir, las librerías \textit{Primefaces} y \textit{Commons IO}. Para ello, simplemente haremos clic secundario en el directorio \textit{Libraries} del módulo Web (\textit{Booking-war}) y seleccionamos la opción para añadir un archivo JAR (\textit{Add JAR/Folder...}). Seleccionamos una de las dos librerías y haremos seguidamente el mismo proceso para la otra. Observamos que los iconos de errores desaparecen. 

\item Asimismo, observamos que existe una advertencia similar, en este caso se informa que no se encuentra la librería \textit{JavaEE Endorsed API}, una de las dependencias del código. Para solucionarlo, abriremos la ventana de información, como vemos en la figura \ref{fig:advertencia-javaee-endorsed-api} y pulsaremos el botón \textit{Resolve...}. Aquí, añadiremos primeramente una nueva librería pulsando el botón \textit{New Library...} y asignándole el mismo nombre que nos pide: \textit{javaee-endorsed-api-7.0}, en este caso. Una vez creada, se añadirá el archivo \textit{jar} correspondiente a esta librería a través del botón \textit{Add JAR/Folder}. Una vez hecho esto, pulsaremos \textit{OK}.
\end{enumerate}

\begin{figure}
\centering
  \includegraphics[scale=1]{img/instalacion/advertencia-javaee-endorsed-api.jpg}
  \caption{\textit{Advertencia de falta de librería javaee-endorsed-api-7.0}}
  \label{fig:advertencia-javaee-endorsed-api}
\end{figure}

\begin{figure}
\centering
  \includegraphics[scale=.85]{img/instalacion/resolver-javaee-endorsed-api.jpg}
  \caption{\textit{Resolver la dependencia de la librería javaee-endorsed-api-7.0}}
  \label{fig:resolver-javaee-endorsed-api}
\end{figure}

\item Prodeceremos ahora a la instalación de \textit{PostgreSQL} y \textit{pgAdmin} para su uso. Si no se crea un servidor por defecto, crearemos uno usando "\textit{localhost}" para el nombre y el \textit{host} y el resto de campos por defecto, como \textit{5432} para el puerto. 

\item Seguidamente crearemos un usuario dándole el nombre de "\textit{booking}".

\item Por último, crearemos nuestra base de datos "\textit{booking}", seleccionando al usuario con mismo nombre como propietario de la misma. 

\item A continuación realizaremos la carga de los datos de una base de datos básica. Para ello, haciendo clic secundario en la base de datos \textit{booking}, seleccionamos \textit{Restore...} y utilizamos la copia facilitada en el directorio \textit{db}. En el mismo, podemos consultar los datos de acceso para el sistema, así como un pequeño manual para crear nuevos administradores.

\item Antes de realizar la comprobación que todo está bien instalado y configurado, habrá que realizar la configuración del servidor de aplicaciones, en este caso \textit{GlassFish}.

\begin{enumerate}
\item Iniciaremos el servidor a través de \textit{NetBeans}. Para ello, navegaremos hasta la pestaña \textit{Services}, y expandiremos \textit{Servers}. Aquí, encontraremos nuestro servidor, haciendo clic derecho elegiremos la opción \textit{Start} para iniciarlo, como vemos en la figura \ref{fig:iniciar-glassfish}.

\begin{figure}[H]
\centering
  \includegraphics[scale=.95]{img/instalacion/iniciar-glassfish.jpg}
  \caption{\textit{Iniciando GlassFish}}
  \label{fig:iniciar-glassfish}
\end{figure}

\item Una vez finalizado de iniciar, podemos acceder a toda la configuración de \textit{GlassFish} -de acuerdo al archivo \textit{glassfish-resources.xml}, figura \ref{fig:glassfish-resources}- a través de nuestro navegador, haciendo uso del puerto 4848, accediendo mediante la URL: http://localhost:4848/ 
\item Configuraremos primero la conexión del servidor con la base de datos. Para ellos iremos a \textit{Resources > JDBC > JDBC Connection Pools}. Aquí añadiremos una nueva conexión a través del botón \textit{New...}. Aquí, le asignaremos un nombre y elegiremos el tipo de recurso y driver, como se muestra en la figura \ref{fig:new-connection-pool}. Además, se establecerán las propiedades de la base de datos como muestra la figura \ref{fig:connection-pool-properties}, de acuerdo a lo establecido en el archivo \textit{glassfish-resources.xml} que vemos en la imagen \ref{fig:glassfish-resources}.

\begin{figure}
\centering
  \includegraphics[scale=.45]{img/instalacion/new-connection-pool.png}
  \caption{\textit{Creando una nuevo conexión}}
  \label{fig:new-connection-pool}
\end{figure}

\begin{figure}
\centering
  \includegraphics[scale=.45]{img/instalacion/connection-pool-properties.png}
  \caption{\textit{Añadiendo las propiedades de la conexión}}
  \label{fig:connection-pool-properties}
\end{figure}

\begin{figure}
\centering
  \includegraphics[scale=.70]{img/instalacion/glassfish-resources.png}
  \caption{\textit{Archivo glassfish-resources.xml}}
  \label{fig:glassfish-resources}
\end{figure}

\item A continuación, se crea el recurso \textit{JDBC (Java Database Connectivity)}. Para ello, navegaremos a \textit{JDBC Resources}, dentro del mismo nivel del menú, pulsamos el botón para crear uno nuevo, indicamos el nombre y elegimos la conexión creada previamente. 

\begin{figure}
\centering
  \includegraphics[scale=0.5]{img/instalacion/new-jdbc-resource.png}
  \caption{\textit{Creando un nuevo recurso JDBC}}
  \label{fig:new-jdbc-resource}
\end{figure}

\item Finalmente, para acabar la configuración, se especificará las tablas y columnas de la base de datos que se usarán para el control de sesiones de los usuarios en la aplicación. Para ello navegamos a \textit{Configurations > server-config > Security > Realms}. Añadiremos uno nuevo y especificaremos los datos necesarios para el uso del nombre de usuario y contraseña en las sesiones. La figura \ref{fig:new-realm} muestra los valores que usaremos para tal fin, qué tablas y columnas. Como vemos en la imagen, también añadiremos las propiedades necesarias, como hicimos anteriormente en la conexión con la BD.

\begin{figure}
\centering
  \includegraphics[scale=1]{img/instalacion/new-realm.png}
  \caption{\textit{Creando un nuevo ámbito para la seguridad de sesiones}}
  \label{fig:new-realm}
\end{figure}

\end{enumerate}

\item Por último, y antes de ejecutar el programa, añadiremos el archivo \textit{postgresql-9.3-1102.jdbc41.jar} al directorio \textit{glassfish > lib} dentro del directorio de \textit{GlassFish}.

\item Volviendo a \textit{NetBeans}, procedemos a comprobar que todo funciona correctamente compilando el proyecto (clic secundario al proyecto y seleccionamos \textit{Build}) y procediendo al \textit{deploy} (mismo procedimiento, clicando \textit{Deploy}). 

\item Una vez finalizado el proceso con éxito, abriremos el navegador web que usemos en nuestro equipo y accederemos a la aplicación mediante la URL \textit{localhost:8080}. 

\item Si todo ha ido bien, nos aparecerá la página de inicio de sesión. En caso que se produzca algún error podemos consultarlo en la ventana de \textit{GlassFish} del \textit{Output} de nuestro \textit{NetBeans}. Si se trata de un error que no se logre solucionar, puedes contactar con el autor, Jesús Soriano, a través de su web \cite{JesusSoriano}, del repositorio de \textit{GitHub} \cite{github-booking} o mediante correo electrónico: info@jesussoriano.com.
\end{enumerate}


\section{Pruebas de implantación}

Para comprobar que el sistema funciona correctamente, realizaremos pruebas básicas. 

\begin{itemize}
\item Para empezar, se procederá con el inicio de sesión de alguno de los usuarios que se facilitan.
\item A continuación, podemos realizar el registro de nuevos usuarios.
\item Una vez comprobado que esto se realiza correctamente, se puede hacer uso de las funcionalidades del sistema por parte tanto del superadministrador, como administrador y usuario. Como por ejemplo, creación de nuevos servicios o citas, subida de archivos, edición de perfiles, suspensión y activación de servicios y usuarios, reserva de plazas, comprobación de notificaciones, creación de noticias, etc. 
\item El proceso para realizar dichos ejemplos se puede consultar en el manual de usuario (capítulo \ref{sec:manual-usuario}).
\item Se recomienda la creación de usuarios reales para la utilización del sistema y no usar los que se facilitan como ejemplo. 
\item Si tenemos problemas con los datos de la BD, podemos comprobar si la conexión con la misma se hace correctamente a través de la pestaña \textit{Services}, desplegando \textit{Databases} y probando si nuestra base de datos (que debería estar listada al desplegar) conecta adecuadamente haciendo clic secundario y eligiendo la primera opción para conectar.
\end{itemize}


\section{Procedimientos de operación y nivel de servicio}

Si se realizan modificaciones o ampliaciones de requisitos del sistema, se recomienda que se trabaje siempre guardando una copia de seguridad tras la finalización de un nuevo componente o modificación de uno existente. En este PFC se ha trabajado usando \textit{GitHub} como repositorio \textit{Git}. También es recomendable guardar un backup de la base de datos cada cierto tiempo, o cuando los datos son sólidos, por si hubiese algún problema con la misma o se requiera una nueva instalación del sistema. \\

En caso de usarse el sistema en producción, los backups de base de datos serían casi de carácter obligatorio para evitar la pérdida de datos de usuarios reales. 



\chapter{Manual de usuario}
% !TEX encoding = UTF-8 Unicode
% ------------------------------------------------------------------------------
% Este fichero es parte de la plantilla LaTeX para la realización de Proyectos
% Final de Grado, protegido bajo los términos de la licencia GFDL.
% Para más información, la licencia completa viene incluida en el
% fichero fdl-1.3.tex

% Copyright (C) 2012 SPI-FM. Universidad de Cádiz
% ------------------------------------------------------------------------------

Las instrucciones de uso del sistema se detallan a continuación.

\section{Introducción}

Este es un sistema de gestión para un centro deportivo, desarrollado concretamente para \textit{CoreSport}, centro para la mejora de la salud y el rendimiento. \\

La aplicación web será accesible desde cualquier dispositivo con conexión a internet y un navegador web, distinguiéndose 3 tipos de usuarios: superadministrador, administrador y usuario. Los socios de la empresa, y trabajadores si se estima oportuno, serán los administradores, mientras que los usuarios serán los clientes del centro. El rol de superadministrador será llevado a cabo por la persona encargada del sistema, en este caso el propio alumno desarrollador del proyecto. 


\section{Características}

Este sistema de gestión proporciona numerosas características que a continuación se detallan: 

\begin{itemize}
\item Los usuarios, administradores y superadministradores podrán realizar las siguientes acciones: 

\begin{itemize}
\item Seleccionar idioma.
\tiem Registrarse en el sistema\footnote{Todo nuevo registro se dará de alta como "usuario", si se tratase de un administrador, será asignado como tal por el superadministrador u otro administrador}.
\tiem Iniciar y cerrar sesión.
\tiem Recuperar la contraseña en caso de olvido.
\tiem Cambiar su contraseña y el resto de sus datos del perfil.
\tiem Mandar y leer correo interno.
\tiem Ver notificaciones del sistema.
\tiem Reservar plaza en una clase y cancelar las reservas.
\tiem Solicitar citas de algún servicio específico, así como cancelar la solicitud o reserva de cita.
\tiem Consultar las reservas realizadas.
\tiem Ver el calendario de actividades con todas las disponibles, las pasadas y las reservadas.
\tiem Ver el histórico de acciones realizadas en el sistema.
\tiem Ver los comunicados.
\tiem Descargarse los archivos a los que tenga acceso. 
\end{itemize}

\item Respecto a administradores y superadministardores, además de estas funcionalidades, podrán: 

\begin{itemize}
\item Responder a solicitudes de cita.
\item Cancelar la cita de un usuario.
\item Activar, suspender y editar usuarios.
\item Activar o suspender a otro administrador.
\item Ver el histórico de acciones de los usuarios del sistema y otros administradores.
\item Dar de alta, editar, suspender o activar servicios.
\item Dar de alta, editar, suspender o activar clase.
\item Dar de alta, editar, suspender o activar cita.
\item Crear nuevo, editar, asignar destinatarios o eliminar archivo.
\item Crear nuevo, editar, suspender o activar comunicado.
\end{itemize}

\item El superadministrador, además, podrá:

\begin{itemize}
\item Activar, suspender o editar administradores.
\end{itemize}

\end{itemize}


\section{Requisitos previos}

Para la utilización del sistema no se requiere ningún elemento hardware o software fuera de lo común. Cualquier dispositivo con conexión a internet y navegador web puede hacer uso de ella. 


\section{Uso del sistema}

El sistema no precisa de conocimiento fuera de lo común en un sistema de gestión. La interfaz es intuitiva y las funcionalidades están estructuradas de manera sencilla a través del menú. Se irá relatando cómo realizar las tareas disponibles, mostrando visualmente algunos ejemplos más representativos. \\

El primer paso para usar la aplicación web sería el registro de usuario. Para ello, 










Describir todos los aspectos necesarios para una utilización efectiva y eficiente del sistema por parte de los usuarios.





\chapter{Conclusiones}
% !TEX encoding = UTF-8 Unicode
% ------------------------------------------------------------------------------
% Este fichero es parte de la plantilla LaTeX para la realización de Proyectos
% Final de Grado, protegido bajo los términos de la licencia GFDL.
% Para más información, la licencia completa viene incluida en el
% fichero fdl-1.3.tex

% Copyright (C) 2012 SPI-FM. Universidad de Cádiz
% ------------------------------------------------------------------------------

En este último capí­tulo se detallan las lecciones aprendidas tras el desarrollo del presente proyecto y se identifican las posibles oportunidades de mejora sobre el software desarrollado.

\section{Objetivos alcanzados}

Tras la finalización de este Proyecto Fin de Carrera se han alcanzado tanto objetivos previamente definidos como experiencias motivadoras de las que se hablará en la siguiente sección. \\ 

En pocas semanas desde el inicio del proyecto se alcanzó unos de los objetivos del mismo, la página web pública de la empresa \ref{bio:CoreSport}. Este primer objetivo fue a la vez un aporte de motivación, ya que se obtuvieron los primeros resultados visibles y un pequeño logro personal al finalizar mi segunda página web en activo en ese momento. Desde su realización, la web ha tenido un buen funcionamiento, con resultados visibles para la empresa. \\

Tras el desarrollo y finalización de la parte principal del proyecto el resultado obtenido es bastante más funcional y motivador. Se consigue exitosamente una solución al problema planteado en la introducción de esta memoria \ref{sec:introduccion}, obteniendo un software capaz de gestionar la totalidad de servicios del centro, con la opción de añadir actividades, clases, citas, archivos, usuarios, la comunicación entre ellos, seguimiento de acciones, etc. \\

Se obtiene, por tanto, un sistema que palia todos los objetivos definidos con la aprobación de los clientes finales y el incentivo de haber realizado un completo sistema de gestión que servirá de uso para, al menos, una empresa en expansión con entre 100 y 200 clientes hasta el momento.


\section{Lecciones aprendidas}

La realización de este proyecto me ha aportado muchas cosas interesantes, desde la adquisición de conocimiento respecto a las tecnologías usadas o la mejora en la codificación del software -tanto estructuralmente como en la programación en general- hasta el crecimiento personal a la hora de la resolución de problemas, autoaprendizaje y gestión del tiempo. Aunque no ha sido un camino fácil, claro está. \\

Desde el primer momento, la realización de este proyecto ha sido todo un reto. Para empezar, la decisión de qué lenguajes usar, frameworks, tecnologías, etc. Afortunadamente, mi experiencia laboral e inquietud por el aprendizaje me ha facilitado el trabajo en muchos aspectos, ya que anteriormente al inicio del PFC había estado trabajando con aplicaciones webs usando JavaEE y los frameworks usados. Además, toda la parte de web pública, interfaz de usuario y estilos ha coincidido con, puede decirse que, mis inicios en el aprendizaje de diseño web más formalmente, variante en la que estoy centrando mi carrera profesional actualmente. \\

Aunque la planificación temporal de toda la realización del PFC se estimara para 8 meses, diversas circunstancias han influido en que el espacio temporal se haya alargado hasta los 33 meses, cuatro veces más de lo estimado. Esto no quiere decir que los requisitos hayan crecido en número o dificultad, o que la programación se haya complicado más de lo estimado, sino que, principalmente, el atraso se ha debido a compaginar el desarrollo del proyecto con diversos trabajos a tiempo parcial, desde programador Java hasta diseñador web, incluyendo otros trabajos esporádicos relacionados con el diseño o el arte. Aparte, claro está, de imprevistos que han surgido en el camino, vida social, voluntariado en un grupo Scout, práctica de deporte, etc. Por supuesto, han surgido dificultades de programación -no sería un proyecto real sin algún que otro quebradero de cabeza- y algún cambio en los requisitos por parte de los clientes.\\ 

Aún así, ha sido una gran satisfacción haber acabado este sistema de gestión, habiendo aportado y mejorado competencias en cuanto a este ámbito se refiere.  


\section{Trabajo futuro}

Aunque el sistema de gestión obtenido cumpla con los objetivos y necesidades de la empresa en cuestión, a mi parecer, y en parte consecuencia de cambios de requisitos por parte de la propia empresa, existen diversas mejoras que se pueden aplicar de cara al futuro: 

\begin{itemize}
\item Para empezar, en lo que la gestión de actividades y citas se refiere, una posible mejora, que seguramente se lleve a cabo en un futuro próximo, podría ser la opción de crear clases o citas recursivas. Es decir, que se repitan en el tiempo, ya sea mediante la elección de los días de la semana en las que se repetiría, realizando el calendario de clases o citas semanalmente, o hacerla recursiva para que se repita la misma clase o cita cada semana el mismo día a la misma hora, creando dicha clase o cita solo una vez, eligiendo fecha de finalización del bucle si fuese oportuno.

De aquí derivarían también los grupos de actividades. La empresa ha decidido gestionar todos los grupos de actividades de manera paralela al software, por lo que la aplicación no recoge la gestión de las clases continuadas de actividades como entrenamiento funcional con o sin TRX, pilates, saco búlgaro, etc. Dependiendo del éxito del programa entre los clientes y su respuesta y feedback, se tomará la decisión de gestionarlos también a través de la aplicación o seguir usándola solo para los objetivos marcados actualmente.
\item Otra posible mejora sería la adaptación total del sistema para su uso por parte de distintas empresas. El desarrollo ha sido realizado teniendo en cuenta esta opción, pero habría que completarlo con una gestión completa de organizaciones, y que todo lo relacionado con cada una de ellas esté automatizado, añadiendo simplemente su logo, estilo, etc. Que, por otra parte, sería un trabajo de poco esfuerzo, ya que como se ha mencionado la programación se ha realizado de esta forma, por lo que habría que dedicarle tiempo a la realización de pruebas con varias organizaciones y mejora o ampliación de las funcionalidades que necesiten.
\item También se ha tenido en cuenta una mejora en cuanto a opciones y permisos. Con esto quiero decir, añadir la opción de que cada usuario elija si desea que se le pueda contactar mediante el correo interno, si desea recibir notificaciones por correo, etc. Esta mejora se aplicará próximamente, seguramente antes de que pase a producción. 
\item Otra de las posibles tareas a implementar sería la segmentación de usuarios. Es decir, la agrupación de los mismos dependiendo de unas ciertas características para, por ejemplo, destinar archivos subidos o mostrar comunicados solo a un grupo de usuarios, dependiendo si hacen un tipo concreto de actividad, usen algún servicio específico o que cumplan unos criterios de edad o peso.
\item Una mejora que se tiene en cuenta para futuras versiones del software es la de personalización de la interfaz. En este momento es posible personalizarla a través del archivo de la hoja de estilo, pero se podría añadir la opción de customizar los colores de elementos como el cabecero, botones, etc, o elección del color principal y secundario.
\item Un requisito que no se ha citado por parte de los clientes ha sido la internacionalización de la web pública. Por motivos de marketing y usabilidad, esta característica también será tenida muy en cuenta para un futuro próximo.
\item Añadir otros idiomas para la interfaz sería otra opción a introducir con relativa facilidad. Solo habría que traducir el archivo de propiedades del lenguaje al idioma deseado. 
\item Y por último, respecto a la misma web pública, se está barajando la posibilidad de realizar un nuevo diseño usando \textit{WordPress} \ref{WordPress} un \textit{CMS} \ref{CMS} mundialmente conocido y cada vez más extendido en el diseño web. Este cambio daría acceso a los clientes a la gestión de la web para realizar acciones puntuales, como la escritura de entradas del blog, por ejemplo. También facilitaría el mantenimiento de la web, así como posibles extensiones de la web, como la creación de un módulo de ventas. 
\end{itemize}




\chapter*{\bibname}
\addcontentsline{toc}{chapter}{\bibname}
%\renewcommand{\bibname}{}

%\input{./bibliografia}

\begingroup
  \def\chapter*#1{}
\renewcommand{\bibname}{}
% Bibliografí­a con BibTeX
\bibliographystyle{apalike}
\bibliography{bibliografia}

\backmatter

\input{./anexos/general}
%GNU \input{./anexos/fdl-1.3}

\end{document}
